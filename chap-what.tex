This survey serves as an introduction and detailed overview of some topics in (probability) distribution testing, an area of theoretical computer science which falls under the general umbrella of \emph{property testing}, and sits at the intersection of computational learning, statistical learning and hypothesis testing, information theory, and (depending whom one asks) the theory of machine learning. Broadly speaking, distribution testing is concerned with the following type of questions: 
\begin{framed}
 Given a \textbf{small} number of independent data points from some blackbox random source, can we \textbf{efficiently} decide whether the distribution of the data follows some purported model (``property''), or is statistically far from doing so?
\end{framed}

\tbc

\paragraph{A note.} For simplicity, throughout this survey we will sweep under the rug many measure-theoretic subtleties, and assume probability distributions, probability density functions (pdf), and probability mass functions (pmf) exist whenever required, and are suitably well-behaved. We will also typically identify a probability distribution with its pdf or pmf, and by a slight abuse of notation use $\p$ indifferently for the distribution itself and its pdf. Most, if not all, of those subtleties can be handled by inserting the words ``Radon--Nikodym,'' ``measurable,'' and ``counting measure'' in suitable places and order.

\section{Formulation, and relation to Hypothesis Testing}


In what follows, $\ab\in\N$ will be used to parametrize the domain of the probability distributions: namely, $\distribs{\ab}$ will denote the set of probability distributions over a (known) domain $\domain_\ab$.

We begin with the notion of distance we will be concerned about, the total variation distance (also known as \emph{statistical distance}):
\begin{definition}[Total variation distance]
  \label{def:tv}
  The \emph{total variation distance} between two probability distributions $\p,\q\in\distribs{\ab}$ is given by
  \[
    \totalvardist{\p}{\q} = \sup_{S\subseteq \domain_\ab} (\p(S)-\q(S))\,.
  \]
  Given a subset $\class\subseteq\distribs{\ab}$ of distributions, we further define the distance from $\p\in\distribs{\ab}$ to $\class$ as $\totalvardist{\p}{\class} \eqdef \inf_{\q\in\class} \totalvardist{\p}{\q}$, and will say that $\p$ is \emph{$\dst$-far from $\class$} if $\totalvardist{\p}{\class}\geq \dst$.
\end{definition}
One can check that $\dtv$ defines a metric on $\distribs{\ab}$, and takes values in $[0,1]$.\exercise{Check it!} Moreover, the total variation distance exhibits several important properties, some of which will be detailed at length in~\cref{app:distances}; we recall a crucial one below.

\begin{fact}[Data Processing Inequality]
  \label{fact:dpi}
  Suppose $X$ and $Y$ are independent random variables with distributions $\p$ and $\q$, and let $f$ be any (possibly randomized) function independent of $X,Y$. Then the probability distributions $\p_f$ and $\q_f$ of $f(X)$ and $f(Y)$ satisfy
  \[
      \totalvardist{\p_f}{\q_f} \leq \totalvardist{\p}{\q}\,.
  \]
  That is, \emph{postprocessing cannot increase the total variation distance.}
\end{fact}


Assuming that $\p,\q$ are absolutely continuous with respect to some dominating measure $\mu$,
  \begin{equation}
    \totalvardist{\p}{\q} = \frac{1}{2}\int \abs{\dv{\p}{\mu}-\dv{\q}{\mu}}\dd{\mu}
  \end{equation}
  In the discrete case where $\p,\q$ are both over $\N$ or a finite domain, this leads to
  \begin{equation}
    \label{eq:tv:l1}
    \totalvardist{\p}{\q} = \frac{1}{2}\normone{\p-\q}
  \end{equation}
  that is, ``total variation is half the $\lp[1]$ distance between pmfs.'' This turns out to be a very useful connection, since $\lp[p]$ norms are quite well-studied beasts: we get to use our arsenal of geometric inequalities --- H\"older, Cauchy--Schwarz, and monotonicity of $\lp[p]$ norms, to name a few.\smallskip

%nondecreasing in their first argument and nonincreasing in the last two
One last piece of terminology: a \emph{property}\index{property} of distributions is a predicate we are interested in (\eg ``is the probability distribution unimodal?''). By identifying the predicate with the set of objects which satisfy it, we can equivalently view a property of distributions as a \emph{subset} of probability distributions (typically, with some interesting structure). Which is what we will do: throughout, a property is just an arbitrary subset of distributions we are interested in. With this in hand, we are ready to provide a formal definition of what a ``testing algorithm'' is.
\begin{definition}[Testing algorithm]
  \label{def:testing}
Let $\property= \bigcup_{\ab=1}^\infty \property_\ab$ and $\class= \bigcup_{\ab=1}^\infty \class_\ab$ be two properties of probability distributions, where $\property_\ab, \class_\ab\subseteq \distribs{\ab}$ for all $\ab$; and $\ns\colon\N\times(0,1]\times(0,1]\to\N$, $\tc\colon\N\times(0,1]\times(0,1]\to\N$ be two functions. A \emph{testing algorithm for \property under $\class$ with sample complexity $\ns$ and time complexity $\tc$} is a (possibly randomised) algorithm $\Algo$ which, on input $\ab\in\N,\dst \in(0,1], \errprob\in(0,1]$, and a multiset $S$ of $\ns(\ab,\dst,\errprob)$ elements of $\domain_\ab$, runs in time at most $\tc(\ab,\dst,\errprob)$ and outputs $\mathbf{b} \in\{\reject,\accept\}$ such that the following holds.
\begin{itemize}
  \item If $S$ is \iid from some $\p\in\property_\ab$, then $\probaDistrOf{S,\Algo}{\mathbf{b}=\accept} \geq 1-\errprob$;
  \item If $S$ is \iid from some $\p\in\class_\ab$ such that $\totalvardist{\p}{\property_\ab} \geq \dst$, then $\probaDistrOf{S,\Algo}{\mathbf{b}=\reject} \geq 1-\errprob$,
\end{itemize}
where in both cases the probability is over the draw of the \iid sample $S$ from the (unknown) $\p$, and the internal randomness of $\Algo$.
\end{definition}
A few remarks are in order. First, in most of our applications we will take $\class_\ab=\distribs{\ab}$, so that the unknown distribution $\p$ is \emph{a priori} arbitrary, and the goal is to check whether it belongs to the subset (property) of interest $\property_\ab$. However, this need not always be the case, and we may want to choose $\class_\ab$ differently to perform hypothesis testing \emph{under structural assumptions}: for instance, to test whether an unknown unimodal distribution is actually Binomial (in this case, $\property_\ab \subsetneq \class_\ab\subsetneq \distribs{\ab}$), or if say a log-concave distribution is monotone (in which case there is no inclusion relation between $\property_\ab$ and $\class_\ab$, and both are strict subsets of $\distribs{\ab}$).

Another important point is that, while our main focus will be on \emph{discrete} distributions,~\cref{def:testing} allows for continuous distributions as well. Finally, the above definition is quite flexible, and can be seen to allow for testing \emph{multiple} distributions: for instance, taking $\domain_\ab=[\ab]\times[\ab]$, $\class_\ab \eqdef \setOfSuchThat{ \p\in\distribs{\ab} }{ \p=\p_1\otimes\p_2 }$ (product distributions), and $\property_\ab \eqdef \setOfSuchThat{\p_1\otimes\p_2 \in \class_\ab}{\p_1=\p_2}$, we obtain the question of two-sample testing (a.k.a.\ closeness testing), which asks to test whether two unknown distributions over $[\ab]$ are equal, or far from each other.

\paragraph{Dependence on the error probability $\errprob$.} Our definition of testing algorithm leaves the error probability $\errprob$ as a free parameter; however, it is quite common in the distribution testing literature to set it as some arbitrary constant smaller than $1/2$ (usually $1/3$). Indeed, by a standard argument,\index{probability amplification} an error probability $1/3$ can be driven down to arbitrary $\errprob$ at the price of a $\bigO{\log(1/\errprob)}$ factor in the sample complexity. 
\begin{lemma}[Error probability amplification]
  \label{lemma:error:proba:amplification}
  Fix $\property$ and $\class$, and suppose there exists a testing algorithm $\Algo$ for $\property$ under $\class$ with sample complexity $\ns(\ab,\dst,1/3)$ and time complexity $\tc(\ab,\dst,1/3)$. Then there is a testing algorithm $\Algo'$ for $\property$ under $\class$ with sample and time complexities $\ns'(\ab,\dst,\errprob) \eqdef \ns(\ab,\dst,1/3)\clg{18\ln(1/\errprob)}$ and $\tc'(\ab,\dst,\errprob) \eqdef \bigO{ \tc(\ab,\dst,1/3) \log(1/\errprob)}$.\cmargin{Logarithm is binary!}
\end{lemma}
\begin{proof}[Proof sketch]
Fix $\property$, $\class$, $\Algo$ as in the statement. Given $\ab,\dst$, and $\errprob\in(0,1]$, let $\Algo'$ be the algorithm which takes as input a multiset of $\ns'(\ab,\dst,\errprob)$ elements, partitions it (arbitrarily) in $m \eqdef \clg{18\ln(1/\errprob)}$ disjoint multisets $S_1,\dots, S_m$, runs $\Algo$ independently on those $m$ multisets with error probability $1/3$ to get $\mathbf{b}_1,\dots,\mathbf{b}_m$, and finally outputs the majority answer $\mathbf{b} \eqdef \indic{\sum_{i=1}^m \mathbf{b}_i \geq m/2}$. The running time is dominated by the $m$ executions, giving the claimed $\bigO{m\cdot \tc(\ab,\dst,1/3)}$ bound. Thus, it suffices to check that the output is correct with probability at least $1-\errprob$; this in turn follows from a Hoeffding bound (\cref{theo:hoeffding}). Indeed, by assumption, each $\mathbf{b}_i$ is independently correct with some probability $p\geq 2/3$. Letting $X_i \sim \bernoulli{p}$ be the indicator of the event ``{$\mathbf{b}_i$ is the correct output},'' we have
\[
  \probaOf{\mathbf{b} \text{ incorrect}} = \probaOf{ \frac{1}{m}\sum_{i=1}^m X_i  < \frac{1}{2} } \leq e^{-2(p-1/2)^2m} \leq e^{-m/18} \leq \errprob\,,
\]
where we used our setting of $m$ in the last inequality. 
\end{proof}
Importantly, this logarithmic dependence is not always the right one: as we will see in~\cref{chap:identity}, there exist natural problems for which the right dependence on the error probability only scales as $\sqrt{\log(1/\errprob)}$.

%%%%%%%%%%%%%%%%%%%%%%%%%%%%%%%%%%%%%%%%%%%%%%%%%%%%%%%%
%%%%%%%%%%%%%%%%%%%%%%%%%%%%%%%%%%%%%%%%%%%%%%%%%%%%%%%%
\section{Why total variation distance?}

The standard formulation of distribution testing, as stated in~\cref{def:testing}, is tied to a specific metric between probability distribution: the total variation distance (\cref{def:tv}). It is natural to wonder of that choice is arbitrary, and, if not, what motivates it.

\tbc
\begin{itemize}
  \item Very stringent -- strong guarantee
  \item Well-behaved: actual metric (triangle inequality), bounded
  \item Data processing inequality (\cref{fact:dpi}): robust to postprocessing
  \item Relation to hypothesis testing
\end{itemize}

% Relation to hypothesis testing
\begin{lemma}[Pearson--Neyman]
  Any (possibly randomized) statistical test which distinguishes between $\p_0$ and $\p_1$ from a single sample must have Type~I (false positive) and Type-II (false negative) errors satisfying
  \[
      \text{Type~I} + \text{Type~II} \geq 1- \totalvardist{\p_0}{\p_1}
  \]
  Moreover, this is achieved by the test which outputs $1$ if, and only if, the sample belongs to the set $S^\ast \eqdef \setOfSuchThat{x}{\p_1(x) > \p_0(x)}$.
\end{lemma}
\begin{proof}
Fix any test $\Algo$ distinguishing between two distributions $\p_0$ and $\p_1$, given a single observation. Letting $\alpha$ and $\beta$ denote the Type~I and Type-II errors of $\Algo$, we have
\begin{align*}
  \alpha+\beta 
  &= \probaDistrOf{\p_0,R}{\Algo(X,R)=1} + \probaDistrOf{\p_1,R}{\Algo(X,R)=0} \\
  &= \shortexpect_{R}[ \probaDistrOf{\p_0}{\Algo(X,R)=1} ] + \shortexpect_{\Algo}[ \probaDistrOf{\p_1}{\Algo(X,R)=0} ] \\
  &= \shortexpect_{R}[ \probaDistrOf{\p_0}{\Algo(X,R)=1} + \probaDistrOf{\p_1}{\Algo(X,R)=0} ]
%   \totalvardist{\p}{\q}
\end{align*}
where we denote by $R$ the internal randomness of $\Algo$. Since, for any fixed realization $r$ of this randomness $R$, the resulting test $\Algo(\cdot,r)$ is deterministic, we can define for any $r$ the \emph{acceptance region} $S_{\Algo,r} \eqdef \setOfSuchThat{x}{\Algo(x,r)=1}$, and write
\begin{align*}
  \alpha+\beta 
  &= \shortexpect_{R}[ \probaDistrOf{\p_0}{X \in S_{\Algo,R}} + \probaDistrOf{\p_1}{X \notin S_{\Algo,R}} ] \\
  &= 1+\shortexpect_{R}[ \p_0(S_{\Algo,R}) - \p_1(S_{\Algo,R}) ] \\
  &\geq 1 + \inf_{S}(\p_0(S) - \p_1(S)) \\
  &= 1 - \sup_{S}(\p_1(S) - \p_0(S)) \\
  &= 1- \totalvardist{\p_0}{\p_1}\,,
\end{align*}
as claimed. Finally, it is immediate from the definition of total variation distance that the proposed test satisfies $\text{Type~I} + \text{Type~II} = 1 + \p_0(S^\ast) - \p_1(S^\ast) = 1- \totalvardist{\p_0}{\p_1}$.
\end{proof}
\tbc
%%%%%%%%%%%%%%%%%%%%%%%%%%%%%%%%%%%%%%%%%%%%%%%%%%%%%%%%
%%%%%%%%%%%%%%%%%%%%%%%%%%%%%%%%%%%%%%%%%%%%%%%%%%%%%%%%
\section{The road not taken: tolerant testing}
\tbc
%%%%%%%%%%%%%%%%%%%%%%%%%%%%%%%%%%%%%%%%%%%%%%%%%%%%%%%%
%%%%%%%%%%%%%%%%%%%%%%%%%%%%%%%%%%%%%%%%%%%%%%%%%%%%%%%%
\section{Historical notes}
\tbc
