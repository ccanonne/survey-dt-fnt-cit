%%%% IMPORTANT: to generate the nomenclature
%% makeindex main-survey-fnt.nlo -s nomencl.ist -o main-survey-fnt.nls


% !Mode:: "TeX:DE:UTF-8:Main"
%
%
%JOURNAL CODE  SEE DOCUMENTATION
\documentclass[biber,plain]{nowfnt} % creates the journal version, needs biber version
%\documentclass[biber]{nowfnt} % creates the journal version, needs biber version
%wrapper for book and ebook are created automatically.
\usepackage{imakeidx}
\usepackage[utf8]{inputenc}

%\usepackage{draftwatermark}
%\SetWatermarkLightness{0.95}

\def\withcolors{1}
\def\withnotes{1}
\usepackage{ccanonne}

\usetikzlibrary{shadows}
\usepackage{framed}
\usepackage{tabularx}
\usepackage[nocfg]{nomencl}
\makenomenclature

% Footnote without marker
\newcommand\extrafootertext[1]{%
    \bgroup
    \renewcommand\thefootnote{\fnsymbol{footnote}}%
    \renewcommand\thempfootnote{\fnsymbol{mpfootnote}}%
    \footnotetext[0]{#1}%
    \egroup
}

%%%%%%%%%%%%%%%%% Exercises %%%%%%%%%%%%%%%%%
\newtheorem{question}{Exercise}[chapter]

%\newcommand{\exinline}[1]{(\refstepcounter{question}Exercise~\thequestion\label{#1})}

\crefname{question}{Exercise}{Exercises}
\Crefname{question}{Exercise}{Exercises}
\crefname{claim}{Claim}{Claims}

\newtheorem{solution}{Solution}[chapter]
%%%%%%%%%%%%%%%%%%%%%%%%%%%%%%%%%%%%%%%%%%%%

%BIBLIOGRAPHY FILE
\addbibresource{bibliography.bib}

\newcolumntype{Y}{>{\centering\arraybackslash}X}
\renewcommand\tabularxcolumn[1]{m{#1}}% for vertical centering text in X column


\usepackage{physics} %% <- redefines var
\usepackage[fulladjust]{marginnote}
\renewcommand*{\marginfont}{\scriptsize}
\setlength{\marginparwidth}{15mm}
\newcommand{\exercise}[1]{\marginnote{\textbf{E:} #1}}

\ifnum\withnotes=1
	\newcommand{\tbc}{\noindent\hl{\sc{}to be continued}\xspace}
\else
	\newcommand{\tbc}{}
\fi


\ifnum\withcolors=1
	\newcommand{\nss}{\textcolor{red}{\ns_1}}
	\newcommand{\nst}{\textcolor{red}{\ns_2}}
	\newcommand{\nsu}{\textcolor{red}{\ns_3}}
	\newcommand{\occur}{\textcolor{red}{N}}
	\newcommand{\freq}{\textcolor{purple}{F}}
\else
	\newcommand{\nss}{\ns_1}
	\newcommand{\nst}{\ns_2}
	\newcommand{\nsu}{\ns_3}
	\newcommand{\occur}{N}
	\newcommand{\freq}{F}
\fi


%% Subsections in the ToC (set to 1 for sections only)
\setcounter{tocdepth}{2}

%ARTICLE TITLE
\title{Topics and Techniques in Distribution Testing}


%ARTICLE SUB-TITLE
\subtitle{A Biased but Representative Sample}


%AUTHORS FOR COVER PAGE 
% separate authors by \and, item by \\
% Don't use verbatim or problematic symbols.
% _ in mail address should be entered as \_
% Pay attention to large mail-addresses ...

%if there are many author twocolumn mode can be activated.
%\booltrue{authortwocolumn} %SEE DOCUMENTATION
\maintitleauthorlist{
Cl\'ement L. Canonne \\
University of Sydney\\
clement.canonne@sydney.edu.au
}

%ISSUE DATA AS PROVIDED BY NOW
\issuesetup
{%
 copyrightowner={A.~Heezemans and M.~Casey},
 volume        = xx,
 issue         = xx,
 pubyear       = 2018,
 isbn          = xxx-x-xxxxx-xxx-x,
 eisbn         = xxx-x-xxxxx-xxx-x,
 doi           = 10.1561/XXXXXXXXX,
 firstpage     = 1, %Explain
 lastpage      = 18
 }

%BIBLIOGRAPHY FILE
\addbibresource{bibliography.bib}
\addbibresource{citation.bib}

% \usepackage{mwe}

%AUTHORS FOR ABSTRACT PAGE
\author[1]{Cl\'ement L. Canonne}

\affil[1]{University of Sydney; clement.canonne@sydney.edu.au}

\articledatabox{\nowfntstandardcitation}

\begin{document}
\makeabstracttitle
\begin{abstract}
Solutions to the exercises from~\citet{CIT-114}.
\end{abstract}

\setcounter{chapter}{1}
%%%%%%%%%%%%%%%%%%%%%%%%%%%%%%%%%%%%%%%%%%%%%%%%%%%%%%%%%%%%%%
%%%%%%%%%%%%%%%%%%%%%%%%%%%%%%%%%%%%%%%%%%%%%%%%%%%%%%%%%%%%%%
\chapter{Testing goodness-of-fit of univariate distributions}
  \label{chap:identity}
  
  
 \begin{question}\label{ex:identity:monotonicity:lp}
  Prove the monotonicity of $\lp[p]$ norms: if $1\leq r\leq s \leq \infty$, then $\norm{x}_s \leq \norm{x}_r$ for every $x\in \R^n$.
\end{question}
\begin{solution}
Fix any $1\leq r\leq s < \infty$, and any non-zero $x\in\R^n$ (if $x$ is the zero vector, the inequality is trivially true). Then, since $x' \eqdef x/\norm{x}_s$ has unit $\lp[s]$ norm and $|x'_i| \leq 1$ for all $i$, we get
\[
	1 = \sum_{i=1}^n |x'_i|^s \leq \sum_{i=1}^n |x'_i|^r = \norm{x'}_r^r
\]
showing that $\norm{x}_s^r \leq \norm{x}_r^r$. Taking the $r$-th root on both side gives the result. Finally, the case $s=\infty$ follows from observing that $\norminf{x} = \max_{1\leq i\leq n} |x_i|^r \leq \sum_{i=1}^n |x_i|^r = \norm{x}_r^r$. 
\end{solution}

\begin{question}\label{ex:expectation:z2} 
  Prove Eq. (2.14): that is, the ``unique elements'' statistic $Z_2$ from~Section~2.1.3 has expectation $\bE{\p}{Z_2} = \sum_{i\in\domain} \p(i)(1-\p(i))^{\ns-1}$.
\end{question}
\begin{solution}
By linearity of expectation,
\[
\bE{\p}{Z_2} = \frac{1}{\ns}\sum_{i\in\domain} \probaOf{N_i = 1}
\]
where $N_i = \sum_{t=1}^\ns \indic{X_t=i}$ follows a Binomial distribution with parameters $\ns$ and $\p(i)$. Thus, $\probaOf{N_i = 1} = \ns \p(i)(1-\p(i))^{\ns-1}$.
\end{solution}
\begin{question}\label{ex:uniformity:moments:poisson}
  Establish Claim~2.2, using (or computing) the expression for the first 4 moments of a $\poisson{\lambda}$ random variable.
\end{question}
\begin{solution}
Let $\lambda,\mu\geq 0$, and $X\sim\poisson{\lambda}$. Then $(X-\mu)^2 = (X-\lambda)^2 + 2(\lambda-\mu)X + \mu^2-\lambda^2$, and so
\begin{align*}
	\bEE{(X-\mu)^2 - X} &= \Var[X] - \bEE{X} + 2(\lambda-\mu)\bEE{X} + \mu^2-\lambda^2 \\
	&= 2(\lambda-\mu)\lambda + \mu^2-\lambda^2
	= (\lambda-\mu)^2
\end{align*}
using the fact that $\Var[X] = \bEE{X} = \lambda$. For the second one, we will also use the identities
\begin{align*}
\bEE{X^2} &= \Var[X] + \bEE{X}^2 = \lambda+\lambda^2\\
\bEE{X^3} &= \lambda+3\lambda^2+\lambda^3\\
\bEE{X^4} &= \lambda+7\lambda^2+6\lambda^3+\lambda^4
\end{align*}
(which are not hard to prove by manipulating the corresponding series $\bEE{X^m} = e^{-\lambda} \sum_{k=0}^{\infty} \frac{\lambda^{m+k}}{k!}$, but are quite tedious).
Then, a brute-force computation gives
\begin{align*}
	&\bEE{((X-\mu)^2 - X)^2} \\
	&\;= 
	\mu^4+\bEE{X^4} - (4\mu+2)\bEE{X^3} + (6\mu^2+4\mu+1)\bEE{X^2} - (4\mu^3+2\mu^2)\bEE{X} \\
	&\;= 
	\mu^4+\lambda^4+6\lambda^3+7\lambda^2+\lambda
	- (4\mu+2)(\lambda^3 + 3\lambda^2 + \lambda) \\
	&\qquad+ (6\mu^2+4\mu+1)(\lambda^2+\lambda) - (4\mu^3+2\mu^2)\lambda \\
	&= \textcolor{red}{\mu^4}+\textcolor{red}{\lambda^4}+4\lambda^3+\textcolor{blue}{2\lambda^2} - \textcolor{red}{4\mu\lambda^3} - 8\mu\lambda^2 + \textcolor{red}{6\mu^2\lambda^2}+6\mu^2\lambda - \textcolor{red}{4\mu^3\lambda}-2\mu^2\lambda \\
	&= \textcolor{red}{(\lambda-\mu)^4} + \textcolor{blue}{2\lambda^2}
	+4\lambda^3- 8\mu\lambda^2+4\mu^2\lambda  \\
	&= \textcolor{red}{(\lambda-\mu)^4} + \textcolor{blue}{2\lambda^2}
	+4\lambda(\lambda-\mu)^2
\end{align*}
which is the result we wanted. There probably are more elegant ways to prove it, but this one works.
\end{solution}
\begin{question}\label{ex:uniformity:bias:coin}
  Establish the upper bound part of Fact~2.1, by proving \textit{via} an Hoeffding or Chernoff bound that the empirical estimator achieves the stated sample complexity. (The lower bound can be shown by considering the case $\alpha=1/2$, but we have not seen in this chapter the information-theoretic tools to establish it: this will be in Chapter 3)
\end{question}
\begin{solution}
Let $X_1,\dots, X_\ns \sim \bernoulli{\alpha}$ be \iid, and consider the empirical estimator (for $\alpha$), 
\[
	\hat{\alpha} \eqdef \frac{1}{\ns}\sum_{i=1}^\ns X_i.
\]
By linearity of expectation, we have $\bEE{\hat{\alpha}} = \frac{1}{\ns}\sum_{i=1}^\ns \bEE{X_i} = \alpha$. Moreover, by a Hoeffding bound (Corollary~A.4), we have, for any $\eta>0$,
\[
	\bPr{\abs{\hat{\alpha} - \alpha}> \eta} \leq 2e^{-2\eta^2 \ns}
\]
which is at most $\errprob$ for
 $\ns \geq \frac{1}{2\eta^2}\ln\frac{2}{\errprob}$. Thus, having $\ns \eqdef \clg{\frac{1}{2\eta^2}\ln\frac{2}{\errprob}} = \bigO{\frac{\log(1/\errprob)}{\eta^2}}$ suffices.

\end{solution}
\begin{question}\label{ex:uniformity:bias:coin:testing}
  Establish the upper bound part of Fact~2.2, by proving \textit{via} a Chernoff bound that appropriately thresholding the empirical estimator achieves the stated sample complexity. (For the lower bound, same remark as for~\cref{ex:uniformity:bias:coin}.)
\end{question}
\begin{solution}
Let $X_1,\dots, X_\ns \sim \bernoulli{\alpha}$ be \iid, $\beta,\eta\in(0,1]$, and consider as before the empirical estimator
\[
	\hat{\alpha} \eqdef \frac{1}{\ns}\sum_{i=1}^\ns X_i
\]
along with the threshold $\tau \eqdef (1+\frac{\eta}{2})\beta$. We want to argue that, for $\ns$ as in the statement of the exercise:
\begin{itemize}
	\item If $\alpha \leq \beta$, then $\bPr{\hat{\alpha} \geq \tau} \leq \errprob$; and
	\item If $\alpha \geq \beta(1+\eta)$, then $\bPr{\hat{\alpha} < \tau} \leq \errprob$.
\end{itemize}
By a Chernoff bound (specifically, Theorem~A.6, (A.7) with $\gamma \eqdef \eta/2\in(0,1]$ and $P_H \eqdef \ns\beta$), in the first case we have
\[
	\bPr{\hat{\alpha} \geq \tau} \leq e^{-\frac{\ns\eta^2\beta}{12}}\,
\]
while in the second case (by (A.8), with $\gamma \eqdef \frac{\eta}{2(1+\eta)}\in(0,1]$ and $P_L \eqdef \ns(1+\eta)\beta$, so that $(1-\gamma)P_L = \ns(1+\eta/2)\beta$) we get
\[
	\bPr{\hat{\alpha} < \tau} \leq e^{-\frac{\ns\eta^2\beta}{4(1+\eta)^2}}
	\leq e^{-\frac{\ns\eta^2\beta}{16}}\,.
\]
Both are at most $\errprob$ as long as $\ns \geq \frac{16}{\beta\eta^2}\ln \frac{1}{\errprob}$. Thus, having $\ns \eqdef \clg{\frac{16}{\beta\eta^2}\ln\frac{1}{\errprob}} = \bigO{\frac{\log(1/\errprob)}{\beta\eta^2}}$ suffices.
\end{solution}
\begin{question}\label{ex:uniformity:bipartite}
  Follow the analysis of Theorem~2.1 to derive, for the bipartite collisions tester, the guarantee Eq.~(2.38) from the variance bound Eq.~(2.37).
\end{question}
\begin{solution}
We have
\begin{align}
  \Var[Z_6] 
  &\leq \frac{1}{\nss\nst}\normtwo{\p}^2 + \frac{\nss+\nst}{\nss\nst}(\norm{\p}_3^3-\normtwo{\p}^4) \tag{2.37}\,,
\end{align}
and we want to show that, in the ``far'' case,
\begin{align}
  \probaDistrOf{\p}{ Z_6 < \frac{1+2\dst^2}{\ab} }
  \leq \frac{5\ab}{4\dst^4\nss\nst} + \frac{\nss+\nst}{\nss\nst}\Paren{\frac{2\sqrt{\ab}}{\dst}+\frac{3}{\dst^2}} \tag{2.38}\,.
\end{align}
We will mimic the corresponding part of the proof of Theorem~2.1:
%%%%%%%%%%%%%%%%%%%%%%%%%%%%%%%%%%%%%%%%%%%%%%%%%%%%%%%%%%%%%%%%%%%%%
Let again $\alpha^2 \eqdef \ab\normtwo{\p-\uniform_\ab}^2 \geq 4\dst^2$, so that $\bEE{Z_6} = \normtwo{\p}^2 = \frac{1+\alpha^2}{\ab}$. Then
    \begin{align*}
      \bPr{Z_6 < \frac{1+2\dst^2}{\ab}} 
      &= \bPr{Z_6 < \frac{1+2\dst^2}{1+\alpha^2}\bEE{Z_6}} \\
      &= \bPr{Z_6 < \Paren{1-\frac{\alpha^2-2\dst^2}{1+\alpha^2}}\bEE{Z_6}} \\
      &\leq \bPr{Z_6 < \Paren{1-\frac{\alpha^2}{2(1+\alpha^2)}}\bEE{Z_6}} \tag{as $\alpha^2 \geq 4\dst^2$}\\
      &\leq \frac{4(1+\alpha^2)^2}{\alpha^4}\cdot \frac{\Var[Z_6]}{\bEE{Z_6}^2} \tag{Chebyshev}\\
      &\leq \frac{4(1+\alpha^2)^2}{\alpha^4\nss\nst\normtwo{\p}^2} + \frac{4(1+\alpha^2)^2(\nss+\nst)}{\alpha^4\nss\nst}\cdot \frac{\norm{\p}_3^3-\normtwo{\p}^4}{\normtwo{\p}^4} 
    \end{align*}
    the last inequality using (2.37). The first term is easily dealt with: recalling that $\normtwo{\p}^2 = (1+\alpha^2)/\ab$,
    \[
        \frac{4(1+\alpha^2)^2}{\alpha^4\nss\nst\normtwo{\p}^2} = \frac{4(1+\alpha^2)\ab}{\alpha^4\nss\nst} \leq \frac{5\ab}{4\dst^4\nss\nst}
    \]
    the last inequality as in the proof of Theorem~2.1, using that $x>0 \mapsto \frac{1+x}{x^2}$ is decreasing, $\alpha^2\geq 4\dst^2$, and $\dst \leq 1$.
    
    To handle the second, we use the inequality proven in (2.12):
    \begin{align*}
        \norm{\p}_3^3-\normtwo{\p}^4
        &\leq \frac{\alpha^{3}}{\ab^{3/2}} + \frac{3\alpha^2}{\ab^2}
    \end{align*}
    which as in (2.13) implies
    \begin{align*}
        \frac{4(1+\alpha^2)^2}{\alpha^4}\cdot \frac{\norm{\p}_3^3-\normtwo{\p}^4}{\normtwo{\p}^4}
        %&= \frac{4\ab^2}{\alpha^4}\cdot (\norm{\p}_3^3-\normtwo{\p}^4) \notag\\
        %&\leq \frac{4\sqrt{\ab}}{\alpha}+ \frac{12}{\alpha^2} \notag\\
        &\leq \frac{2\sqrt{\ab}}{\dst}+ \frac{3}{\dst^2}\,
    \end{align*}
    Combining the two, we get
    \[
        \probaDistrOf{\p}{ Z_6 < \frac{1+2\dst^2}{\ab} } \leq \frac{5\ab}{4\dst^4\nss\nst} + \Paren{\frac{3\sqrt{\ab}}{\dst} +\frac{3}{\dst^2\ns}}\frac{\nss+\nst}{\nss\nst}
    \]
    as we wanted.

%%%%%%%%%%%%%%%%%%%%%%%%%%%%%%%%%%%%%%%%%%%%%%%%%%%%%%%%%%%%%%%%%%%%%
\end{solution}
\begin{question}\label{ex:nomallet}
  Show that, in contrast to what we did in the empirical-distance tester case (Section 2.1.5), one cannot invoke stochastic dominance in the analysis of the bipartite collision tester to obtain the wishful variance bound Eq.~(2.39) instead of Eq.~(2.41). Specifically, show that it fails even for $\ab=2$: if $M\sim\binomial{\nss}{p}$, $N\sim\binomial{\nst}{p}$ and $M'\sim\binomial{\nss}{q}$, $N'\sim\binomial{\nst}{q}$ (all independent) with $1/2\leq q < p \leq 1$, it is \emph{not} always true that
  \[
      MN+(\nss-M)(\nst-N) \succeq M'N'+(\nss-M')(\nst-N')
  \]
  \emph{Hint: consider the case $\nss=1$, and $\bPr{MN+(\nss-M)(\nst-N)\geq 1}$ as a function of $p$.}
\end{question}
\begin{solution}
If the stochastic dominance relation held, it would imply that $f(p) \eqdef \bPr{MN+(\nss-M)(\nst-N)\geq 1}$ is a non-decreasing function of $p\geq 1/2$. Now, to simplify the search for a counterexample, consider the case $\nss=1$ (so $M\sim\bernoulli{p}$): then
\begin{align*}
	f(p) 
	&= \bPr{MN+(1-M)(\nst-N)\geq 1} \\
	&= \bPr{M = 1, N > 0 \text{ or } M=0, N < \nst } \\
	&= p(1-(1-p)^{\nst}) + (1-p)(1-p^{\nst})
\end{align*}
using independence of $M,N$. One can try to plot the corresponding function of $p$ for various choices of $\ns_2$, or differentiate to check if it is non-decreasing on $[1/2,1]$. Long story short: it will be non-decreasing for $\ns_2\in\{1,2,3\}$, but for $\ns_2=4$, we get
\begin{align*}
	f(p) &= p(1-(1-p)^4) + (1-p)(1-p^4)
\end{align*}
which has a local minimum at $p^\ast=\frac{3+\sqrt{3}}{6}$, is decreasing on $[1/2, p^\ast]$ and increasing on $[p^\ast, 1]$. 

\begin{tikzpicture}
    \begin{axis}[xlabel=$p$, ylabel={$f(p)=\bPr{MN+(1-M)(4-N)\geq 1}$}]
        \addplot[domain=0.5:1] {x*(1-(1-x)^4) + (1-x)*(1-x^4)};
    \end{axis}
\end{tikzpicture}

This gives a counterexample to the statement for, \eg $p=1/2, q=\frac{3+\sqrt{3}}{6}$, $\ns_1=1$ and $\ns_2=4$.
\end{solution}
\begin{question}\label{ex:identity:averaging:stochastic} 
    It is known that $x\preceq y$ if, and only if, $x=Ay$ for some doubly stochastic matrix $A$~\citep[Theorem~2.1]{Arnold87}. Check that the averaging from Lemma~2.7 indeed corresponds to multiplying the pmf $\p$ (seen as a vector) by such a matrix.
\end{question}
\begin{solution}
Recall that a doubly stochastic matrix is a square matrix with non-negative entries, where each row and each column sums to one. In our case, assume for simplicity that the probability distribution $\p$ is non-decreasing, \ie that $\p(1)\geq \dots \geq(\ab)$. This is without loss of generality, since one can permute the domain for this to hold, and doubly stochastic matrices are invariant to such permutations: if $\sigma$ is a permutation of $[\ab]=\{1,2,\dots,\ab\}$ and $A$ is doubly stochastic, then for every $j\in[\ab]$
\[
	\sum_{i=1}^\ab A_{\sigma(i),\sigma(j)} = \sum_{i=1}^\ab A_{i,\sigma(j)} = 1
\]
and similarly for the columns sums: for every $i\in[\ab]$, $\sum_{j=1}^\ab A_{\sigma(i),\sigma(j)} = \sum_{j=1}^\ab A_{\sigma(i), j} = 1$. Now, with this assumption, then the transformation to obtain $\bar{\p}$ is to average the probability of the first $K=\clg{\ab/2}$ elements, and leave the remaining $\ab-K$ probabilities unchanged. This is achieved by the matrix $A\in\R^{\ab\times\ab}$ consisting of a square $K\times K$ block with all entries equal to $1/K$ in the top left, and the rest being only $\ab-K$ diagonal entries equal to $1$:
\[
	A = 
	\begin{pmatrix}
	\frac{1}{K} & \cdots & \frac{1}{K} & 0 &\cdots & 0\\
	\vdots & \ddots & \vdots & 0 &\cdots & 0\\
	\frac{1}{K} & \cdots & \frac{1}{K} & 0 &\cdots & 0\\
	0 & \cdots & 0 & 1 &\cdots & 0\\
	\vdots & \cdots & \vdots & 0 &\ddots & 0\\
	0 & \cdots & 0 & 0 &\cdots & 1\\
	\end{pmatrix} =
	\begin{pmatrix}
	\frac{1}{K} \mathbf{1}_{K\times K} & \mathbf{0}_{K\times (\ab-K)} \\
	\mathbf{0}_{(\ab-K)\times K} & {I}_{(\ab-K)\times (\ab-K)} \\
	\end{pmatrix}
\]
It is now easy to check that the matrix $A$ defined above is indeed doubly stochastic, and further that
\[
	A \p = 	\begin{pmatrix}
	\frac{1}{K} & \cdots & \frac{1}{K} & 0 &\cdots & 0\\
	\vdots & \ddots & \vdots & 0 &\cdots & 0\\
	\frac{1}{K} & \cdots & \frac{1}{K} & 0 &\cdots & 0\\
	0 & \cdots & 0 & 1 &\cdots & 0\\
	\vdots & \cdots & \vdots & 0 &\ddots & 0\\
	0 & \cdots & 0 & 0 &\cdots & 1\\
	\end{pmatrix} 
	\begin{pmatrix}
	\p(1)\\
	\vdots \\
	\p(K)\\
	\p(K+1)\\
	\vdots\\
	\p(\ab)
	\end{pmatrix} 
	= \begin{pmatrix}
	\frac{1}{K}(\p(1)+\dots \p(K))\\
	\vdots \\
	\frac{1}{K}(\p(1)+\dots \p(K))\\
	\p(K+1)\\
	\vdots\\
	\p(\ab)
	\end{pmatrix} 
	= \bar{\p}
\]
as we wanted.
\end{solution}
\begin{question}[$\star$]\label{ex:linftycheck}
  Generalize Lemma~2.13 to relax the condition $\nsu \leq \ab^{2/3}$ to $\nsu \leq \ab^{(s-1)/s}$, for any fixed (constant) integer $s\geq 3$, by considering $s$-collisions instead of $3$-collisions in Algorithm~8. How does the $\lp[\infty]$ guarantee bound in (ii) change with $s$?
\end{question}
\begin{solution}
Fix any $s\geq 4$ (Lemma~2.13 already handled $s=3$). Recalling (2.42), we get that under the uniform distribution $\uniformOn{\ab}$ the probability to observe an $s$-way collision among $\ns$ samples is at most
\[
	p(\ns,\ab,s) \leq \frac{1}{\ab^{s-1}}\binom{\ns}{s} \leq \frac{1}{\ab^{s-1}}\Paren{\frac{\ns}{s}}^s
\]
which is at most $1/6$ for $\ns \leq \frac{\ab^{1-1/s}}{6^{1/s}s}$. Now, let $C=C(s)$ be a value to be determined in the course of the analysis, and assume that $\norminf{\p}> \frac{C}{\ns}$, so that as in the proof of Lemma~2.13, fixing an arbitrary $i$ such that $\p(i) = \norminf{\p}$, the number of times $i$ appears among the $\ns$ samples, $\occur_i$, is Binomially distributed with parameters $\ns$ and $\norminf{\p}$, and thus mean $\ns\norminf{\p} > C$. Again by a Chernoff bound (specifically, (A.8)), we have that the probability \emph{not} to observe an $s$-way collision on element $i$ is at most, choosing $\gamma$ such that $(1-\gamma) C=s$ and recalling that $s \geq 4$,
\[
	\probaOf{\occur_i < s } = \probaOf{\occur_i < (1-\gamma) C } \leq e^{-\gamma^2 C/2} = e^{-\frac{\gamma^2}{2(1-\gamma)}s} \leq e^{-\frac{2\gamma^2}{1-\gamma}}
\]
which is less than $1/6$ for (solving numerically), \eg $\gamma \geq 0.82$. From the above, this means that we can take any $C \geq \frac{s}{1-\gamma}$ (so for instance $C = 6s$ suffices).

Putting the two conditions together, what we get is an algorithm which, for any fixed $s\geq 4$, distinguishes  with probability at least $5/6$ between (i)~$\p=\uniformOn{\ab}$ and (ii)~$\norminf{\p}\geq \frac{6s}{\ns}$, provided that $\ns \leq \frac{\ab^{1-1/s}}{6^{1/s}s}$. The algorithm does so by checking if any $s$-way collision happens among the $\ns$ samples, and declaring (i) if, and only if, no such collision is observed.\medskip

\noindent As an example, for $s \asymp\ln \ab$, we get the following:
\begin{corollary}
There exists an algorithm which, given $\ns$ \iid samples from some unknown $\p\in\distribs{\ab}$,  distinguishes  with probability at least $5/6$  between (i)~$\p=\uniformOn{\ab}$ and (ii)~$\norminf{\p}\geq \frac{\ln \ab}{\ns}$, provided that $\ns \lesssim \frac{\ab}{\ln \ab}$. 
\end{corollary}
\end{solution}

\begin{question}[$\star$]
  Recall that our $\chi^2$-based statistic (Eq. (2.19)) was analyzed under the Poissonized sampling model, which led us to define it with a $-\occur_i$ term in the numerator. We will show that this term is necessary: that is, under the Poissonization assumption, consider the ``simpler'' statistic
  \[
      Z'_3 \eqdef \sum_{i=1}^\ab \frac{(\occur_i-\ns/\ab)^2}{\ns/\ab}\,.
  \]
  Show that its expectation is $\ns\ab\normtwo{\p-\uniform_\ab}^2 +\ab$ (so the expectation \emph{gap} remains the same), but that the variance now contains an extra term $\frac{\ab^2}{\ns}$. What sample complexity does this yield?
\end{question}
\begin{solution}
By linearity of expectation, we can just use the same analysis (and the expression established in~\cref{ex:uniformity:moments:poisson}), ``adding back'' the $\bEE{\occur_i} = \ns\p(i)$ terms, to get
\begin{align*}
      \bEE{Z'_3} 
      &= \sum_{i=1}^\ab \frac{\bEE{(\occur_i-\ns/\ab)^2 - \occur_i} + \bEE{\occur_i}}{\ns/\ab} \\
      &= \frac{\ab}{\ns}\sum_{i=1}^\ab \Paren{\Paren{\ns\p(i)-\frac{\ns}{\ab}}^2 + \ns\p(i)} \\
      &= \ns\ab \sum_{i=1}^\ab \Paren{\p(i)-\frac{1}{\ab}}^2 + \ab \sum_{i=1}^\ab \p(i) \\
      &= \ns\ab\normtwo{\p-\uniform_\ab}^2 +\ab\,,
\end{align*}
so, indeed, the expectation gap $\bE{\p}{Z'_3} - \bE{\uniformOn{\ab}}{Z'_3} = \ns\ab\normtwo{\p-\uniform_\ab}^2$ remains the same. However, the variance is now (using independence of the summands)
\begin{align*}
      \Var[Z'_3] 
      &= \sum_{i=1}^\ab \frac{\Var[(\occur_i-\ns/\ab)^2]}{\ns^2/\ab^2} \\
      &= \frac{\ab^2}{\ns^2}\sum_{i=1}^\ab \Paren{\bEE{(\occur_i-\ns/\ab)^4} - \bEE{(\occur_i-\ns/\ab)^2}^2}
\end{align*}
At this point, we have to compute this horrible-looking expression. This is quite painful, but as in~\cref{ex:uniformity:moments:poisson} (or using something like Mathematica if one would rather not go through that ordeal again) one can check that, for $X\sim \poisson{\lambda}$ and any $\mu$,
\begin{equation}
	\bEE{(X-\mu)^4} - \bEE{(X-\mu)^2}^2 = 2\lambda^2 + 4\lambda(\lambda-\mu)^2 + 4\lambda(\lambda-\mu) + \lambda\,. 
\end{equation}
This leads to
\begin{align*}
      \Var&[Z'_3] \\
      &= \frac{\ab^2}{\ns^2}\sum_{i=1}^\ab \Paren{2\ns^2\p (i)^2 + 4\ns^3\p(i)\Paren{\p(i)-\frac{1}{\ab}} ^2 + 4\ns^2\p(i)\Paren{\p(i)-\frac{1}{\ab}} + \ns\p(i)} \\
      &= \Var[Z_3] + \frac{\ab^2}{\ns^2}\sum_{i=1}^\ab \Paren{4\ns^2\p(i)\Paren{\p(i)-\frac{1}{\ab}} + \ns\p(i)} \\
      &= \Var[Z_3] + 4\ab^2\sum_{i=1}^\ab \p(i)\Paren{\p(i)-\frac{1}{\ab}} + \frac{\ab^2}{\ns}\\
      &= \Var[Z_3] + 4\ab^2\Paren{\normtwo{\p}^2 - \frac{1}{\ab}} + \frac{\ab^2}{\ns} \\
      &= \Var[Z_3] + 4\ab^2\normtwo{\p - \uniformOn{\ab}}^2 + \frac{\ab^2}{\ns}
\end{align*}
where we recognized (and took out) the expression of $\Var[Z_3]$ from Section~2.1.4. To see the sample complexity this would lead to, note that to get ``variance $\ll$ (expectation gap)$^2$'' we will need, considering the three terms of $\Var[Z'_3]$ above,
\begin{align*}
		\Var[Z_3] &\ll (\ns\ab\normtwo{\p-\uniform_\ab}^2)^2,  \\
		\ab^2\normtwo{\p - \uniformOn{\ab}}^2 &\ll (\ns\ab\normtwo{\p-\uniform_\ab}^2)^2,  \\
		\frac{\ab^2}{\ns} &\ll (\ns\ab\normtwo{\p-\uniform_\ab}^2)^2
\end{align*}
The first one is exactly what we had in Section~2.1.4, and leads to the condition $\ns \gg \sqrt{\ab}/\dst^2$. The second is not too problematic: after simplication, and recalling that in the ``far'' case we have $\normtwo{\p-\uniform_\ab} \geq 2\dst/\sqrt{\ab}$, this results in the (weaker) condition $\ns \gg \sqrt{\ab}/\dst$. The third one, however, is the bottleneck: again using the bound on $\normtwo{\p-\uniform_\ab}$ in the ``far'' case (the only handle we have on this quantity), it results in the condition
\[
		\frac{\ab^2}{\ns} \ll \ns^2\ab^2 \cdot \frac{\dst^4}{\ab^2}
\]
which means, reorganizing, that $\ns$ must satisfy
$
	\ns \gg \frac{\ab^{2/3}}{\dst^{4/3}}\,.
$ Altogether, and once made formal via, \eg Chebyshev's inequality as usual, these 3 conditions will yield the sample complexity 
\[
		\ns = \bigO{ \max\Paren{\frac{\ab^{2/3}}{\dst^{4/3}},\frac{\sqrt{\ab}}{\dst^2}} }
\]
which is suboptimal (and, as a side note, is the expression for the sample complexity of \emph{closeness} testing, the harder testing problem where both $\p$ and $\q$ are unknown).
\end{solution}
\begin{question}[$\star$]\label{ex:unif:adaptive}
Combine the doubling search technique discussed in Section~1.1 with the sample complexity of uniformity testing given in Eq.~(2.50) to prove the following. There is an adaptive uniformity testing algorithm which, on input $\ab$ and $\dst\in(0,1]$, and access to samples from an unknown distribution $\p\in\distribs{\ab}$:
\begin{itemize}
  \item correctly distinguishes between (1)~$\p=\uniform_\ab$ and (2)~$\dst(\p)\eqdef \totalvardist{\p}{\uniform_\ab} > \dst$, with probability at least $2/3$;
  \item always takes at most 
    \[
      \bigO{\frac{1}{\dst^2}\Paren{\sqrt{\ab \log \log \frac{1}{\dst} } + \log \log \frac{1}{\dst} }}
    \] samples; but also
  \item if $\dst(\p) > \dst$, takes at most
    \[
      \bigO{\frac{1}{\dst(\p)^2}\Paren{\sqrt{\ab \log \log \frac{1}{\dst(\p)} } + \log \log \frac{1}{\dst(\p)} }}
    \] samples, with probability at least $2/3$; and, finally,
  \item show that this constant-probability bound on the number of samples also holds \emph{in expectation.}
\end{itemize}
That is, in the ``far'' case this algorithm never does much worse (up to a $\log\log$ factor) than an ideal algorithm provided with the exact value $\dst(\p)$ and asked to distinguish between $\p=\uniform_\ab$ and $\totalvardist{\p}{\uniform_\ab} = \dst(\p)$.
\end{question}
\begin{solution}%[Outline]
As discussed in Section~1.1, the overall idea is to run a uniformity tester sequentially, with varying parameters (the $j$-th instance, for $0\leq j\leq L$, being run with parameters $\ab, \dst_j, \errprob_j$, for decreasing values of $\dst_j$), and to stop and return $\reject$ if any of the tester's invocations returns \reject. If all of the $L+1$ invocations returns $\accept$, then we return \accept.

Specifically, we set $L\eqdef \clg{\log(1/\dst)}$, and for $0\leq j\leq L$ choose
\[
	\dst_j \eqdef 2^{-j}, \qquad \errprob_j \eqdef \frac{2}{\pi^2(j+1)^2}
\]
If the unknown distribution $\p$ \emph{is} uniform, then by a union bound all $L+1$ invocations of the uniformity tester will return $\accept$ with overall probability at least
\[
	1 - \sum_{j=0}^L \errprob_j \geq 1 - \sum_{j=0}^\infty \errprob_j = 1 - \frac{2}{\pi^2}\sum_{j=0}^\infty \frac{1}{(j+1)^2} = \frac{2}{3}\,.
\]
However, if $\dst(\p)\eqdef \totalvardist{\p}{\uniformOn{\ab}} > \dst$, then there exists $1\leq j(\p)\leq L$ such that
\[
	\frac{1}{2^{j(\p)}} < \dst(\p) \leq \frac{1}{2^{j(\p)-1}}
\]
and so either the algorithm rejects before reaching invocation $j(\p)$ (which is alright) or reaches invocation $j(\p)$, when it then rejects with probability at least
\[
	1-\errprob_{j(\p)} = 1-\frac{2}{\pi^2(j(\p)+1)^2} \geq 1-\frac{1}{2\pi^2} \geq \frac{2}{3}.
\]
This deals with the correctness; we still need to establish the 3 components of the sample complexity (worst-case as a function of $\dst$, with high probability as a function of $\dst(\p)$ in the non-uniform case, and on expectation as a function of $\dst(\p)$ in the non-uniform case). Let $\ns(\ab,\dst, \errprob)$ denote the optimal sample complexity from Eq.~(2.50).
\begin{itemize}
\item Since we are running at most $L+1$ invocations of the uniformity tester, the sample complexity is at most the sum of these $L+1$ sample complexities, and so is bounded by
\begin{align*}
	\sum_{j=0}^L \ns(\ab,\dst_j, \errprob_j)
	&\asymp \sum_{j=0}^L  \frac{\sqrt{\ab\log(1/\errprob_j)} + \log(1/\errprob_j)}{\dst_j^2} \\
	&\asymp \sum_{j=0}^L  2^{2j}\Paren{\sqrt{\ab\log(j+1)} + \log(j+1)}\\
	&\asymp 2^{2L}\Paren{\sqrt{\ab\log(L+1)} + \log(L+1)}
\end{align*}
which, recalling our choice of $L=\clg{\log(1/\dst)}$, is
\[
	\bigO{\frac{\sqrt{\ab\log\log(1/\dst)} + \log\log(1/\dst)}{\dst^2}}
\]
as claimed.
\item In the non-uniform case, where $\dst(\p) > \dst$, it suffices to note that with probability at least $1-\errprob_{j(\p)}$ the algorithm with stop at the $j(\p)$ invocation (where $j(\p)$ is as defined above, within a factor two of $\dst(\p)$). We thus can reuse the above analysis of the sample complexity, but stopping at $j(\p)$, to obtain that with probability at least $1-\errprob_{j(\p)} \geq 2/3$ the number of samples taken will be
\[
	\bigO{2^{2j(\p)}\Paren{\sqrt{\ab\log(j(\p)+1)} + \log(j(\p)+1)}}\,,
\]
which is $\bigO{\frac{\sqrt{\ab\log\log(1/\dst(\p))} + \log\log(1/\dst(\p))}{\dst(\p)^2}}$.
\item Finally, to get the same guarantee \emph{on expectation} in the non-uniform case, let $T$ (a random variable) denote the index of the last invocation of the tester, so that $0\leq T \leq L$. We have seen that $T=j(\p)$ with probability at least $1-\errprob_{j(\p)}$; but we have much stronger guarantees! Namely, for any $j > j(\p)$,
\[
	\probaOf{T \geq j} \leq \prod_{i=j(\p)}^{j-1}\errprob_{i} \leq \errprob_{j(\p)}^{j-j(\p)}
\]
since this means that all invocations from $j(\p)$ onwards must have failed (\ie did not reject even though they should have). The expected sample complexity is then at most
\[
	\sum_{j=0}^L \probaOf{T \geq j}\cdot  \ns(\ab,\dst_j, \errprob_j)
	\leq \sum_{j=0}^{j(\p)} \ns(\ab,\dst_j, \errprob_j)
	+ \sum_{j=j(\p)+1}^L \errprob_{j(\p)}^{j-j(\p)} \ns(\ab,\dst_j, \errprob_j)\,.
\]
We have already analyzed the first term of the RHS, showing it was $\bigO{\frac{\sqrt{\ab\log\log(1/\dst(\p))} + \log\log(1/\dst(\p))}{\dst(\p)^2}}$. The second term is at most (ignoring constants and recalling the setting of $\errprob_j$)
\begin{align*}
	\sum_{j=j(\p)+1}^L &\frac{1}{(2\pi^2)^{j-j(\p)}} \cdot 2^{2j}\Paren{\sqrt{\ab\log(j+1)} + \log(j+1)} \\
	&\leq 2^{2j(\p)}\sum_{j=1}^\infty \Paren{\frac{2}{\pi^2}}^{j}\Paren{\sqrt{\ab\log(j+j(\p)+1)} + \log(j+j(\p)+1)} \\
	&\lesssim 2^{2j(\p)}\Paren{\sqrt{\ab\log(j(\p)+1)} + \log(j(\p)+1)}
\end{align*}
which is $\bigO{\frac{\sqrt{\ab\log\log(1/\dst(\p))} + \log\log(1/\dst(\p))}{\dst(\p)^2}}$ as well. Here, we relied on the fact that $\frac{2}{\pi^2} < 1$ to be able to bound the converging series by its first term: at its core, this is possible because the probability $\probaOf{T \geq j}$ to keep running the tests after reaching $j(\p)$ decreases exponentially quickly with $j$.
\end{itemize}
\end{solution}
\begin{question}\label{ex:mixture:sample}
Given two probability distributions $\p,\q$, an integer $\ns\geq 1$, and a parameter $\alpha\in[0,1]$, consider the following two sampling processes:
\begin{itemize}
  \item Sample $\occur \sim \poisson{\ns}$, and draw $\occur$ \iid samples from the mixture $(1-\alpha)\p+\alpha\q$.
  \item Sample $\occur \sim \poisson{\ns}$, and draw $\occur$ \iid samples from $\p$. Then, for each $1\leq i\leq \occur$, independently sample $B_i\sim\bernoulli{\alpha}$: if $B_i=1$, replace the $i$-th sample by a new (and independent from everything else) sample drawn from $\q$.
\end{itemize}
Show that these two processes result in the same distribution.
\end{question}
\begin{solution}[Sketch]
Condition on a given value of $N$. In both cases, the $N$ samples are mutually independent, so it is enough to show that the marginal distribution of a single sample is the same in both cases. This part is then quite straightforward: suppose $X$ is a sample from the mixture $(1-\alpha)\p+\alpha\q$, and $Y$ obtained by the second process (draw independently $Y'\sim\p$, $Y''\sim \q$, and $B\sim\sim\bernoulli{\alpha}$, and set $Y=(1-B)\cdot Y'+B\cdot Y''$). Then, for all $x$,
\[
	\bPr{X=x} =  (1-\alpha)\p(x)+\alpha\q(x)
\] 
while
\begin{align*}
	\bPr{Y=x} 
	&=  \probaCond{Y=x}{B=0}\cdot \bPr{B=0} + \probaCond{Y=x}{B=1}\cdot \bPr{B=1} \\
	&= \probaCond{Y'=x}{B=0}\cdot (1-\alpha) + \probaCond{Y''=x}{B=1}\cdot \alpha \\
	&=  \p(x)(1-\alpha)+\q(x)\alpha\,.
\end{align*}
\end{solution}
\begin{question}[$\star$]\label{ex:l2:closeness}
Establish the analogue of Theorem~2.22 for the \emph{two-distribution} case (when both $\p,\q$ are unknown, and you are given $\ns$ \iid samples from each). Specifically, consider the statistic
$Z'=\sum_{i=1}^\ab \Paren{(X_i-Y_i)^2-X_i-Y_i}$
for which you will have to establish the following counterpart of Claim~2.2:
\begin{claim}
	\label{claim:closeness:moments:poisson}
If $X\sim\poisson{\lambda}$ and $Y\sim\poisson{\mu}$ are independent, then 
  $
  \bEE{(X-Y)^2-X-Y} = (\lambda-\mu)^2
  $
  and 
  $
  \bEE{((X-Y)^2-X-Y)^2} = (\lambda-\mu)^4 + 2(\lambda+\mu)^2 + 4(\lambda+\mu)(\lambda-\mu)^2
  $.
\end{claim}
\noindent Show that the sample complexity is $O(\max\Paren{\normtwo{\p},\normtwo{\q}}/\dst^2)$. Try to establish the (incomparable) bound $O(\min\Paren{\normtwo{\p},\normtwo{\q}}/\dst^2+1/\dst)$.
\end{question}
\begin{solution}
We will not here establish Claim~2.1, which can be checked by tedious computations \emph{(if you have an elegant proof of this claim, let me know!)}. Consider the following algorithm, the analogue of Algorithm~11 for two unknown distributions:
\begin{algorithm}[ht!]
  \begin{algorithmic}[1]
    \Require Multisets of $\ns$ samples each, $x_1,\dots,x_\ns \in \domain$ and $y_1,\dots,y_\ns \in \domain$, parameters $\dst\in(0,1]$.
    \Comment{Assumes Poissonization}
    \State Set $\tau \gets 3\ns^2\dst^2$
    \State Compute
    \[
        Z = \sum_{j\in\domain} \Paren{(\occur_j-\occur_j')^2-\occur_j-\occur_j'}
    \] where $\occur_j \gets \sum_{t=1}^\ns\indic{x_t=j}$, $\occur_j' \gets \sum_{t=1}^\ns\indic{y_t=j}$.
    \If{ $Z \geq \tau$ } \Return \reject \Comment{$\p,\q$ far (in $\lp[2]$)}
    \Else\ 
      \Return \accept \Comment{$\p,\q$ close (in $\lp[2]$)}
    \EndIf
  \end{algorithmic}
  \caption{\label{algo:l2:closeness}\sc Robust $\lp[2]$ Tester (for Closeness)}
\end{algorithm}
The proof will be very similar to that of Theorem~2.22. Assuming Poissonization, if $x_1,\dots,x_\ns$ are independent samples from $\p$ and $y_1,\dots,y_\ns$ from $\q$, we get that, for all $j\in\domain$, $\occur_j\sim\poisson{\ns\p(j)}$  and $\occur'_j\sim\poisson{\ns\q(j)}$ and so, by~\cref{claim:closeness:moments:poisson} above,
\begin{align}
	\bE{\p}{Z} = \sum_{j=1}^\ab (\ns\p(j)-\ns\q(j))^2 = \ns^2 \normtwo{\p-\q}^2
\end{align}
and
\begin{align}
	\Var_{\p}[Z] 
	&= \sum_{j=1}^\ab \Var\mleft[\Paren{(\occur_j-\occur_j')^2-\occur_j-\occur_j'}\mright] \notag\\
	&= \sum_{j=1}^\ab \Paren{ \bEE{((\occur_j-\occur_j')^2-\occur_j-\occur_j')^2} -  \ns^4(\p(j)-\q(j))^4}\notag\\
	&= \sum_{j=1}^\ab \Paren{2\ns^2(\p(j)+\q(j))^2 + 4\ns^3(\p(j)+\q(j))(\p(j)-\q(j))^2}\notag\\
	&\leq 4\ns^2\Paren{\normtwo{\p}^2+\normtwo{\q}^2} + 4\ns^3(\normtwo{\p}+\normtwo{\q})\normtwo{\p-\q}^2
\end{align}
the last step again using $\p(j)+\q(j)\leq \norminf{\p}+\norminf{\q} \leq \normtwo{\p}+\normtwo{\q}$.

\begin{itemize}
	\item If $\normtwo{\p-\q} \leq \dst$, then $\bE{\p}{Z} \leq \ns^2\dst^2$ and by Markov's inequality
	\[
		\bPr{Z \geq 3\ns^2\dst^2 } \leq \frac{\bE{\p}{Z}}{3\ns^2\dst^2} \leq \frac{1}{3}
	\]
	\item If $\normtwo{\p-\q} \geq 2\dst$, then $\bE{\p}{Z} \geq 4\ns^2\dst^2$ and by Chebyshev's 
	\begin{align*}
		\bPr{Z < 3\ns^2\dst^2 } 
		&\leq \frac{16\Var_{\p}[Z]}{\bE{\p}{Z}^2} \leq 
		\frac{64(\normtwo{\p}^2+\normtwo{\q}^2)}{\ns^2\normtwo{\p-\q}^4} + \frac{64(\normtwo{\p}+\normtwo{\q})}{\ns\normtwo{\p-\q}^2} \\
		&\leq \frac{8\max(\normtwo{\p},\normtwo{\q})^2}{\ns^2\dst^4} + \frac{32\max(\normtwo{\p},\normtwo{\q})}{\ns\dst^2}
	\end{align*}
	which is less than $1/3$ for $\ns \geq 100\max(\normtwo{\p},\normtwo{\q})/\dst^2$.
\end{itemize}
This establishes the first bound from the exercise.\medskip

The second is a little trickier, and we here only provide an outline.\footnote{If you have a fully written solution you'd like to include, or if you think something isn't quite right, do contact me!} It will relies on the following lemma (which will not be proven here, but can be shown by, \eg adapting~\citet[Lemma~3.1]{BatuC17}):
\begin{lemma}[{Constant-factor estimate of the $\lp[2]$ norm}]
There exists an algorithm which, given $\ns$ \iid samples from an unknown distribution $\p\in\distribs{\N}$ and a parameter $\tau\in(0,1]$, outputs a value $\hat{\rho}\in(\tau,1]\cup\{\bot\}$ and has the following guarantees:
\begin{itemize}
	\item if $\normtwo{\p} \geq \tau$, then $\hat{\rho} \neq \bot$ with probability at least $9/10$;
	\item if $\hat{\rho} \neq \bot$, then $\normtwo{\p}/2 \leq \hat{\rho} \leq 2\normtwo{\p}$ with probability at least $9/10$;
\end{itemize}
as long as $\ns = \bigTheta{1/\tau}$.
\end{lemma}
Essentially, the above state that it is possible to use $\bigO{1/\tau}$ samples to obtain (with high constant probability) a constant-factor estimate of the $\lp[2]$ norm of a distribution if we are promised that this norm is at least $\tau$ (and, if that norm is much less than $\tau$, then we will detect it).

With this at our disposal, we can proceed as follows: 
\begin{enumerate}
	\item Set $\tau \asymp \dst$, and use the above algorithm with $\bigO{1/\dst}$ samples from both $\p$ and $\q$ to try and get estimates of their $\lp[2]$ norms. 
	\item If we get $\bot$ for both, then we know that $\normtwo{\p-\q} \leq \normtwo{\p}+\normtwo{\q} \leq 2\tau \leq \dst$ and we are done;
	\item If we get estimates $\hat{\rho}_{\p}$, $\hat{\rho}_{\q}\geq \tau$ for both, then 
		\begin{enumerate}
			\item we check that those two values are within a (sufficiently) large constant factor of each other: if not, we know that $\normtwo{\p-\q} \geq 2\dst$ and we are done;
			\item if they are within constant factors, we get that $\normtwo{\p}\asymp \normtwo{\q}$, and so $\min(\normtwo{\p},\normtwo{\q})\asymp \max(\normtwo{\p},\normtwo{\q})$ and we can use the algorithm from the first part, with sample complexity $O(\max\Paren{\normtwo{\p},\normtwo{\q}}/\dst^2)=O(\min\Paren{\normtwo{\p},\normtwo{\q}}/\dst^2)$, and we are done;
		\end{enumerate}
	\item If we get an estimate for one (say $\p$ but $\bot$ for the other (say $\q$), then we know that $\normtwo{\q} \leq \tau$ and have a constant-factor estimate $\hat{\rho}_{\p}$ for $\normtwo{\p}$. 
		\begin{enumerate}
				\item If $\hat{\rho}_{\p} \gg \dst$, then we know that $\normtwo{\p-\q} \geq \normtwo{\p}-\tau \gg 2\dst$ and we are done.
				\item If $\hat{\rho}_{\p} \asymp \dst$, then we know that $\max(\normtwo{\p},\normtwo{\q} \asymp \dst$, so we can use the algorithm from the first part, with sample complexity $O(\max\Paren{\normtwo{\p},\normtwo{\q}}/\dst^2) = O(1/\dst)$, and we are done.
		\end{enumerate}
\end{enumerate}
Overall, the sample complexity is $\bigO{\max\Paren{\min\Paren{\normtwo{\p},\normtwo{\q}}/\dst^2}, 1/\dst}$, establishing the second bound from the exercise.
\end{solution}
\begin{question}\label{ex:l2:reduction:chisquare}
Show that the transformation $\Phi$ from Section~2.2.2 (Eq.~(2.59)) ``maps $\chi^2$ divergence to $\lp[2]$ distance'' in the following, approximate way: for any $\p,\q\in\distribs{\ab}$, 
\[
	\normtwo{\Phi_\q(\p)-\Phi_\q(\q)}^2 = \sum_{i\in\domain}\frac{(\p(i)-\q(i))^2}{1+\flr{\ab\q(i)}\,.}
\]
Conclude that, assuming $\min_i \q(i) \geq 1/(2\ab)$ (as we could in Section~2.2.1 after using the ``mixture trick'' of Eq.~(2.52)), 
\[
	\frac{1}{2}\chisquare{\p}{\q}\leq \ab\normtwo{\Phi_\q(\p)-\Phi_\q(\q)}^2 \leq \chisquare{\p}{\q}
\]
for every $\p\in\distribs{\ab}$.
\end{question}
\begin{solution}
From the expression of $\Phi$ and with the notation of Section~2.2.2, we have
\begin{align*}
	\normtwo{\Phi_\q(\p)-\Phi_\q(\q)}^2
	&= \sum_{j=1}^{\ab'} \Paren{\Phi_\q(\p)-\Phi_\q(\q)}^2 \\
	&= \sum_{j=1}^{\ab'} \Paren{\sum_{i=1}^\ab \Paren{\frac{\p(i)}{\ab_i} - \frac{\q(i)}{\ab_i}} \indic{j\in S_i} }^2\\
	&= \sum_{j=1}^{\ab'} \sum_{i=1}^\ab \indic{j\in S_i}  \Paren{\frac{\p(i)}{\ab_i} - \frac{\q(i)}{\ab_i}}^2 \tag{$\ast$}\\
	&= \sum_{i=1}^\ab \frac{1}{\ab_i^2} \Paren{\p(i)-\q(i)}^2 \sum_{j=1}^{\ab'}\indic{j\in S_i}  \\
	&= \sum_{i=1}^\ab \frac{\Paren{\p(i)-\q(i)}^2}{\ab_i}  \\
\end{align*}
where for $(\ast)$ we used that exactly one term of the inner sum in non-zero (as each $j\in[\ab]$ belongs to exactly one  set $S_i$), allowing us to bring the sum and indicator outside the square; and later that $\sum_{j=1}^{\ab'}\indic{j\in S_i} = \abs{S_i} = \ab_i$. The result then follows from the definition of $\ab_i=1+\flr{\ab\q(i)}$ (for $i\in[\ab]$).

\noindent Finally, if $\min_i \q(i) \geq 1/(2\ab)$ we get that
\[
	\ab\q(i) \leq 1+\flr{\ab\q(i)} \leq 2\ab\q(i), \qquad i\in[\ab]
\]
where the RHS is obtained by observing that $\max_{x\geq 1/2} \frac{1+\flr{x}}{x} = 1/2$. This directly implies
\[
		\underbrace{\frac{1}{2}\sum_{i=1}^\ab \frac{\Paren{\p(i)-\q(i)}^2}{\q(i)}}_{\frac{1}{2}\chisquare{\p}{\q}} \leq \underbrace{\ab \sum_{i=1}^\ab \frac{\Paren{\p(i)-\q(i)}^2}{1+\flr{\ab\q(i)}}}_{\ab\normtwo{\Phi_\q(\p)-\Phi_\q(\q)}^2} 
		\leq \underbrace{\sum_{i=1}^\ab \frac{\Paren{\p(i)-\q(i)}^2}{\q(i)}}_{\chisquare{\p}{\q}}
\]
as wanted.
\end{solution}

\begin{question}[$\star\star$]\label{ex:l1:reduction:parameters}
Generalize the transformation $\Phi$ from Section~2.2.3 in two ways: first, by replacing the mixture $\Phi^{(3)}_\q(\p) = \frac{1}{2}\p+\frac{1}{2}\uniformOn{\ab}$ by $\alpha\p+(1-\alpha)\uniformOn{\ab}$, where $\alpha\in(0,1)$. Second, by replacing the choice $\ab'=4\ab$ in $\Phi^{(2)}(\p)$ by $\ab'=\beta\ab$, for some integer $\beta$ such that $\beta(1-\alpha)\geq 1$.
\begin{enumerate}
  \item By tracking down the various restrictions on $\alpha,\beta$ and their use across $\Phi^{(1)}$, $\Phi^{(2)}$, and $\Phi^{(3)}$, show that doing so now maps identity testing with parameters $(\ab,\dst)$ to uniformity testing with parameters 
  \[
  	\Paren{\beta\ab,\alpha\Paren{1-\frac{1}{\beta(1-\alpha)}}\dst}
  \]
  \item Check that setting $(\alpha,\beta)=(1/2,4)$ as in Section~2.2.3 recovers Theorem~2.28, and the blowup factor of $32$ discussed at the end of the section.
  \item Recalling that the sample complexity scales as $\sqrt{\ab}/\dst^2$, optimize over $(\alpha,\beta)$ to find the optimal choice of parameters, and prove that the resulting blowup is $\approx 12.2$. %% alpha = 4/5, beta=25
  \item What would be the optimal choice of $(\alpha,\beta)$, and the corresponding blowup, in a setting where the sample complexity of uniformity testing scales as $\ab/\dst^2$ instead of $\sqrt{\ab}/\dst^2$? (This is not that far-fetched: we will see in Section~4.3 an example of such a setting.)  %alpha=2/3, beta=9, blowup 46
\end{enumerate}
\end{question}
\begin{solution} The solution below follows the aalysis of Section~2.2.3.
\begin{enumerate}
  \item By setting things as suggested, we will be able to assume in our analysis of $\Phi^{(2)}$ that $\ab'\q(i) \geq \ab'(1-\alpha)/\ab = \beta(1-\alpha)$ for all $i\in[\ab]$, and then
  \begin{equation}
  	\label{eq:exercise:uniformity:reduction:min:qi}
  	\min_i \frac{\flr{\ab'\q(i)}}{\ab'\q(i)} \geq \frac{\beta(1-\alpha)-1}{\beta(1-\alpha)}
  \end{equation}
  (the $\flr{\ab'\q(i)}$ in the numerator is the reason we need to enforce $\beta(1-\alpha)\geq 1$). We then get the analogue of Eq.~(2.65): for every $\p_1,\p_2\in\distribs{\ab}$, 
  \begin{equation}
  			\totalvardist{\Phi^{(2)}(\p_1)}{\Phi^{(2)}(\p_2)} \geq \frac{\beta(1-\alpha)-1}{\beta(1-\alpha)}\totalvardist{\p_1}{\p_2}\,,
  \end{equation}
  We then only have to take into account the last piece, \ie that $\Phi^{(3)}$ does allow us to assume that Eq.~\eqref{eq:exercise:uniformity:reduction:min:qi} holds, but doing so comes at the cost of shrinking the total variation distance by a factor $\alpha$ as well; so that, when combining $\Phi^{(1)}$, $\Phi^{(2)}$, and $\Phi^{(1)}$ all together, we get a mapping $\Phi$ such that, for every $\p_1,\p_2\in\distribs{\ab}$, 
  \begin{equation}
  			\totalvardist{\Phi(\p_1)}{\Phi(\p_2)} \geq \alpha\Paren{1-\frac{1}{\beta(1-\alpha)}}\totalvardist{\p_1}{\p_2}\,.
  \end{equation}
  Overall, for any choice of $\alpha\in[0,1]$ and $\beta \geq 1$ such that $\beta(1-\alpha) \geq 1$, we can convert an identity testing instance with parameters $(\ab,\dst)$ to a uniformity testing instance with parameters $(\ab' = \beta\ab, \dst' = \alpha\Paren{1-\frac{1}{\beta(1-\alpha)}})$.
  \item For $\alpha = 1/2$ and $\beta=4$, we get $\alpha\Paren{1-\frac{1}{\beta(1-\alpha)}} = 1/4$, and so
  \[
  	\frac{\sqrt{\ab'}}{\dst'^2} 
  	= \frac{\sqrt{\beta}}{\Paren{\alpha\Paren{1-\frac{1}{\beta(1-\alpha)}}}^2}\cdot \frac{\sqrt{\ab}}{\dst^2}  
  	= \frac{\sqrt{4}}{(1/4)^2}\cdot \frac{\sqrt{\ab}}{\dst^2} 
  	= \boxed{32}\cdot \frac{\sqrt{\ab}}{\dst^2}
  \]
  \item From the above, we want to choose $\alpha, \beta$ to minimize the factor
  \begin{equation}
  	\label{eq:blowup:factor}
  	\frac{\sqrt{\beta}}{\Paren{\alpha\Paren{1-\frac{1}{\beta(1-\alpha)}}}^2}
  \end{equation}
  subject to $\beta(1-\alpha)\geq 1$, $\alpha\in[0,1]$, $\beta \geq 1$. For the sake of the minimization, it is easier to minimize the square of this quantity, and to set $ \gamma \eqdef \beta(1-\alpha)$, so that we seek to find the minimizer of
  \[
  		\frac{\beta}{\alpha^4\Paren{1-\frac{1}{\beta(1-\alpha)}}^4} = \frac{1}{\alpha^4(1-\alpha)} \cdot \frac{\gamma^5}{(\gamma-1)^4}
  \]
  subject to $\alpha\in[0,1]$ and $\gamma \geq 1$. The variables are now separated, and we can minimize separately in $\alpha$ and in $\gamma$. This leads to minimizers $\alpha^\ast=4/5$ and $\gamma^\ast=5$. Getting back to our original quantity, we therefore get that it is minimized for $(\alpha,\beta)=(4/5, 25)$, for which the blowup factor in Eq.~\eqref{eq:blowup:factor} is approximately $\boxed{12.21}$.
  \item If the sample complexity were to scale as $\ab/\dst^2$ instead of $\frac{\sqrt{\ab}}{\dst^2}$, then the blowup factor would become 
  \begin{equation}
  	\label{eq:blowup:factor:2}
  	\frac{\ab'/\dst'^2}{\ab/\dst^2} = \frac{\beta}{\Paren{\alpha\Paren{1-\frac{1}{\beta(1-\alpha)}}}^2}
  \end{equation}
  and minimizing this, as before subject to $\beta(1-\alpha)\geq 1$, $\alpha\in[0,1]$, $\beta \geq 1$, leads in the same way to minimizers $(\alpha,\beta)=(2/3, 9)$, for which the blowup factor in Eq.~\eqref{eq:blowup:factor:2} is approximately $\boxed{45.56}$.
\end{enumerate}
\end{solution}

%%%%%%%%%%%%%%%%%%%%%%%%%%%%%%%%%%%%%%%%%%%%%%%%%%%%%%%%%%%%%%
%%%%%%%%%%%%%%%%%%%%%%%%%%%%%%%%%%%%%%%%%%%%%%%%%%%%%%%%%%%%%%
\chapter{Information-theoretic lower bounds}
  \label{chap:lowerbounds}
  
\begin{question}\label{exo:deriving:dep:delta:lb}
Combine (the second part of) Lemma~B.4 with (the first part of) Lemma~B.5 to obtain Eq.~(3.16) from Eq.~(3.7). Use it to derive Theorem~3.3.
\end{question}
\begin{solution}
Invoking Lemma~B.4 (specifically, Eq.~(B.9)) and Lemma~B.5, we get
\begin{align*}
	1-2\errprob 
	&\leq \totalvardist{\bE{\theta}{\p_\theta^{\otimes\ns}}}{\p_\accept^{\otimes\ns}} \tag{Eq.~(3.7)}\\
	&\leq 1-\frac{1}{2}e^{-\kldiv{\bE{\theta}{\p_\theta^{\otimes\ns}}}{\p_\accept^{\otimes\ns}}} \tag{Eq.~(B.9)}\\
	&\leq 1-\frac{1}{2(1+\chisquare{\bE{\theta}{\p_\theta^{\otimes\ns}}}{\p_\accept^{\otimes\ns}}} \tag{Lemma~B.5}
\end{align*}
which, reorganizing the inequality, yields Eq.~(3.16):
\[
	\frac{1}{4\errprob} \leq 1+\chisquare{\bE{\theta}{\p_\theta^{\otimes\ns}}}{\p_\accept^{\otimes\ns}}\,.
\]
Now, since in Eq.~(3.14) we had derived 
\[
	\chisquare{\bE{\theta}{\p_\theta^{\otimes\ns}}}{\p_\accept^{\otimes\ns}}
	= \bE{\theta,\theta'}{(1+{H(\theta,\theta')})^\ns} - 1
	\leq e^{\frac{81\dst^4\ns^2}{\ab}} - 1
\]
we can plug this in Eq.~(3.16) to obtain the necessary inequality 
\begin{equation}
\frac{1}{4\errprob} \leq e^{\frac{81\dst^4\ns^2}{\ab}}\,,
\end{equation}
that is, $\ns \geq \frac{ \sqrt{ \ab \log(1/(4\errprob)) } }{ 9\dst^2 }$; proving Theorem~3.3. \emph{(As a side note: one can similarly}\exercise{Try it!} \emph{prove a lower bound of $\ns = \bigOmega{\log(1/\errprob)/\dst^2}$ using Lemma~B.4 instead of Pinsker's inequality in Eq.~(3.4); and combining the two leads to the following bound for uniformity testing, 
\begin{equation}
	\ns = \bigOmega{\frac{ \sqrt{ \ab \log(1/(\errprob)) } + \log(1/\errprob) }{ \dst^2 }}\,,
\end{equation}
which is optimal.}
\end{solution}

\noindent \textcolor{red}{Erratum: The first part of the next exercise (about the distance to $\property_\ab$) was incorrectly stated in the published version.}
\begin{question}\label{exo:notrealproba:lb}
Fix a property $\property_\ab \subseteq \distribs{\ab}$ of distributions, and denote by $\tilde{\property}_\ab$ its ``extension to probability measures'' (not just probability distributions) defined as follows:
\begin{equation}
	\tilde{\property}_\ab \eqdef \setOfSuchThat{ \alpha\q }{ \q\in\property_\ab, \alpha \geq 0 }
\end{equation}
(for instance, for uniformity, $\property_\ab=\{\uniformOn{\ab}\}$ and $\tilde{\property}_\ab = \{\alpha \textbf{1}_\ab\}_{\alpha \geq 0}$.) 
Let $\p$ be a measure (not necessarily a probability measure) such that the $\lp[1]$ distance between $\p$ and $\tilde{\property}_\ab$ satisfies $\lp[1](\p,\tilde{\property}_\ab) > 2\dst$, and $1/2\leq \normone{\p} \leq 3/2$. Defining $\p' \eqdef \p/\normone{\p}$ (an actual probability distribution), provide a lower bound on $\totalvardist{\p'}{\property_\ab}$. Moreover, show that obtaining $\ns$ ``samples'' from the Poisson process with measure $\p$ is equivalent to getting $\poisson{\ns\normone{\p}}$ samples from the distribution $\p'$.

Conclude with how one could use a testing algorithm $\Algo$ for property $\property_\ab$ given $\poisson{\ns}$ samples (\ie in the Poissonized sampling model) to distinguish between two families of measures (\yes- and \no-instances) far in $\lp[1]$ distance, thus justifying the relaxed assumption from Section~3.2.
\end{question}
\begin{solution}
Let $\p$, $\property_\ab$, $\tilde{\property}_\ab$, and $\p'$ as above. Fix any $\q\in \property_\ab$; we have
\begin{align*}
\normone{\p' - \q}
&= \normone{\frac{\p}{\normone{\p}} - \q} 
= \frac{1}{\normone{\p}}\normone{\p - \normone{\p}\q} \\
&\geq \frac{2}{3}\normone{\p - \normone{\p}\q} > \frac{2}{3}\cdot 2\dst
\end{align*}
using for the last inequality that $\normone{\p}\q \in \tilde{\property}_\ab$, as a (positive) rescaling of $\q\in \property_\ab$. As $\q$ was arbitrary, this implies $\totalvardist{\p'}{\property_\ab} > \frac{2}{3}\dst$.

Then, the output of a Poisson process with parameter $\ns\p$ (\ie parameter $\ns$ and underlying measure $\p$) is by definition a set of mutually independent values $\occur_1,\dots, \occur_\ab$, where $\occur_i$ is distributed as $\poisson{\ns\p(i)} = \poisson{\ns\normone{\p}\p'(i)}$. Which is exactly what one get by drawing $\poisson{\ns'}$ \iid samples from $\p'$ for $\ns'\eqdef \ns\normone{\p}$ (not necessarily an integer).

Finally, suppose we have a testing algorithm $\Algo$ for property $\property_\ab$, which succeeds when given $\poisson{\ns}$ samples for $\ns=\ns(\ab, \dst, \errprob)$. Then we can use it to distinguish whether the unknown measure $\p$ is a  \yes- ($\p/\normone{\p}\in \property_\ab$) or a \no- instance ($\p/\normone{\p}$ $2\dst$-far in $\lp[1]$ distance, or equivalently $\dst$-far in TV distance) in our ``relaxed,'' ``Poisson process'' setting by feeding the output of our Poisson process with parameter $\ns'\p$ for $\ns' \eqdef 2\ns(\ab, \frac{2}{3}\dst, \errprob)$. By the above, this corresponds to $\poisson{\ns'\normone{\p}}$ samples from $\p'$, which (i)~in the $\yes$ case is in $\property_\ab$, and (ii)~in \no case will be $\frac{2}{3}\dst$-far from $\property_\ab$, and
\[
	\ns'\normone{\p} = 2\normone{\p} \cdot \ns(\ab, \frac{2}{3}\dst, \errprob) \geq \ns(\ab, \frac{2}{3}\dst, \errprob)
\]
since $\normone{\p} \geq 1/2$; so $\Algo$ will be correct with probability at least $1-\errprob$. This implies that any lower bound in this ``relaxed'' setting carries over, with only constant-factor losses in the parameters, to the Poissonized setting.
\end{solution}

\begin{question}[$\star$]\label{exo:paninski:mi:lb}
Recall that we defined the \no-instances in Section~3.2 by Eq.~(3.17) (measures, instead of \emph{bona fide} probability measures) in order to guarantee mutual independence of $\occur_1,\dots,\occur_\ab$ (conditioned on $\mathbf{b}$. Check the argument to see which part of the argument would fail if we had used Eq.~(3.11) instead. Then, modify the argument to fix this, and obtain the same sample complexity lower bound. \textit{(Hint: we still have mutual independence of the $\ab/2$ random variables $(\occur_1, \occur_2),\dots,(\occur_{\ab-1}, \occur_\ab)$ conditioned on $\mathbf{b}$. Establish the analogue of Eq.~(3.20) with $\occur_1=j,\occur_2=\ell$ instead of $\occur_1=j$, and proceed from there.}
\end{question}
\begin{solution}
We can proceed as in Section~3.2 to bound $\mutualinfo{\mathbf{b}}{X}$, but now keeping in mind (as per the hint) that only the pairs $(\occur_{2i-1}, \occur_{2i})$ are mutually independent conditioned on $\mathbf{b}$, not the $\occur_i$ themselves. We can adapt the argument, and write
\begin{align*}
\mutualinfo{\mathbf{b}}{X} 
&= \entropy{\occur_1,\dots,\occur_\ab} - \entropy{(\occur_1,\dots,\occur_\ab) \mid \mathbf{b}}  \\
&= \entropy{\occur_1,\dots,\occur_\ab} - \sum_{i=1}^{\ab/2} \entropy{(\occur_{2i-1}, \occur_{2i}) \mid \mathbf{b}}  \tag{conditional independence}\\
&\leq \sum_{i=1}^{\ab/2} \entropy{(\occur_{2i-1}, \occur_{2i})} - \sum_{i=1}^{\ab/2} \entropy{(\occur_{2i-1}, \occur_{2i}) \mid \mathbf{b}}  \tag{subadditivity}\\
&= \sum_{i=1}^{\ab/2}\mutualinfo{\mathbf{b}}{(\occur_{2i-1}, \occur_{2i})}\,,
\end{align*}
leading to an analogue of Eq.~(3.19):
\begin{equation}
	\mutualinfo{\mathbf{b}}{X} \leq \frac{\ab}{2} \mutualinfo{\mathbf{b}}{(\occur_{1}, \occur_{2})}\,.
\end{equation}
Building towards the counterpart of Eq.~(3.20), we then have
\begin{align*}
	&\!\!\mutualinfo{\mathbf{b}}{(\occur_{1}, \occur_{2})} \\
	&= \bE{\mathbf{b}}{\kldiv{P_{(\occur_1,\occur_2)\mid \mathbf{b}}}{P_{(\occur_1,\occur_2)}}} \notag\\
	&\leq \bE{\mathbf{b}}{\chisquare{P_{(\occur_1,\occur_2)\mid \mathbf{b}}}{P_{(\occur_1,\occur_2)}}} \notag\\
	&= \frac{1}{2}\Paren{\chisquare{P_{(\occur_1,\occur_2)\mid \mathbf{b}=\accept}}{P_{(\occur_1,\occur_2)}} + \chisquare{P_{(\occur_1,\occur_2)\mid \mathbf{b}=\reject}}{P_{(\occur_1,\occur_2)}}}\notag\\
	&= \frac{1}{2}\Big(\sum_{j=0}^\infty\sum_{\ell=0}^\infty \frac{\Paren{\bPr{\occur_1=j, \occur_2 = \ell\mid \mathbf{b}=\accept }-\bPr{\occur_1=j, \occur_2 = \ell }}^2}{\bPr{\occur_1=j, \occur_2 = \ell}}  \notag\\
	&\qquad+ \sum_{j=0}^\infty\sum_{\ell=0}^\infty \frac{\Paren{\bPr{\occur_1=j, \occur_2 = \ell\mid \mathbf{b}=\reject }-\bPr{\occur_1=j, \occur_2 = \ell }}^2}{\bPr{\occur_1=j, \occur_2 = \ell}} \Big) \notag\\
	&=\frac{1}{2} \sum_{j=0}^\infty\sum_{\ell=0}^\infty \frac{\Paren{\bPr{\occur_1=j, \occur_2 = \ell\mid \mathbf{b}=\reject }-\bPr{\occur_1=j, \occur_2 = \ell\mid \mathbf{b}=\accept }}^2}{\bPr{\occur_1=j, \occur_2 = \ell\mid \mathbf{b}=\reject }+\bPr{\occur_1=j, \occur_2 = \ell\mid \mathbf{b}=\accept }} \notag\\
	%&= \frac{\ab}{2} \sum_{j=0}^\infty\sum_{\ell=0}^\infty \bPr{\occur_1=j, \occur_2 = \ell\mid \mathbf{b}=\accept }\frac{ \Paren{1-  \frac{\bPr{\occur_1=j, \occur_2 = \ell\mid \mathbf{b}=\reject }}{\bPr{\occur_1=j, \occur_2 = \ell\mid \mathbf{b}=\accept }}  }^2}{1+\frac{\bPr{\occur_1=j, \occur_2 = \ell\mid \mathbf{b}=\reject }}{\bPr{\occur_1=j, \occur_2 = \ell\mid \mathbf{b}=\accept }}} \notag\\
	&\leq \frac{1}{2} \sum_{j=0}^\infty\sum_{\ell=0}^\infty \bPr{\occur_1=j, \occur_2 = \ell\mid \mathbf{b}=\accept }\Paren{1-  \frac{\bPr{\occur_1=j, \occur_2 = \ell\mid \mathbf{b}=\reject }}{\bPr{\occur_1=j, \occur_2 = \ell\mid \mathbf{b}=\accept }}  }^2
\end{align*}
using as before for the second-to-last equality that, $\mathbf{b}$ being a uniform bit, $\bPr{\occur_1=j, \occur_2 = \ell} = \frac{1}{2}( \bPr{\occur_1=j, \occur_2 = \ell\mid \mathbf{b}=\accept }+ \bPr{\occur_1=j, \occur_2 = \ell\mid \mathbf{b}=\reject })$ for all $j, \ell\geq 0$.

This gives us~\cref{eq:analogue:320}, our ``new Eq.~(3.20):''
\begin{equation}
	\label{eq:analogue:320}
	\mutualinfo{\mathbf{b}}{X} \leq \frac{\ab}{4} \sum_{j=0}^\infty\sum_{\ell=0}^\infty \bPr{\occur_1=j, \occur_2 = \ell\mid \mathbf{b}=\accept }\Paren{1-  \tfrac{\bPr{\occur_1=j, \occur_2 = \ell\mid \mathbf{b}=\reject }}{\bPr{\occur_1=j, \occur_2 = \ell\mid \mathbf{b}=\accept }}  }^2\,.
\end{equation}

 To bound it, we need to compute $\bPr{\occur_1=j, \occur_2 = \ell\mid \mathbf{b}=\accept }$ and $\bPr{\occur_1=j, \occur_2 = \ell\mid \mathbf{b}=\reject }$ for arbitrary integers $j,\ell\geq 0$. Thankfully, this is not too difficult (recall that we still work in the Poissonized sampling model):
\begin{align*}
\bPr{\occur_1=j, \occur_2 = \ell\mid \mathbf{b}=\accept }
&= e^{-\frac{\ns}{\ab}} \frac{(\ns/\ab)^j}{j!}\cdot e^{-\frac{\ns}{\ab}} \frac{(\ns/\ab)^\ell}{\ell!}
= e^{-\frac{2\ns}{\ab}} \frac{(\ns/\ab)^{j+\ell}}{j!\ell!} \\
\bPr{\occur_1=j, \occur_2 = \ell\mid \mathbf{b}=\reject } \\
&\hspace{-6em}= \frac{1}{2} e^{-\frac{\ns(1+3\dst)}{\ab}} \frac{(\ns(1+3\dst)/\ab)^j}{j!} e^{-\frac{\ns(1-3\dst)}{\ab}} \frac{(\ns(1-3\dst)/\ab)^\ell}{\ell!}  \\
&\hspace{-6em}+ \frac{1}{2} e^{-\frac{\ns(1-3\dst)}{\ab}} \frac{(\ns(1-3\dst)/\ab)^j}{j!} e^{-\frac{\ns(1+3\dst)}{\ab}} \frac{(\ns(1+3\dst)/\ab)^\ell}{\ell!} \\
&\hspace{-6em}=  e^{-\frac{2\ns}{\ab}} \frac{(\ns/\ab)^{j+\ell}}{j!\ell!} \cdot \frac{(1+3\dst)^j(1-3\dst)^\ell+(1-3\dst)^j(1+3\dst)^\ell}{2}\,,
\end{align*}
and so, for all $j,\ell\geq 0$,
\begin{equation}
\frac{\bPr{\occur_1=j, \occur_2 = \ell\mid \mathbf{b}=\reject }}{\bPr{\occur_1=j, \occur_2 = \ell\mid \mathbf{b}=\accept }} = \frac{1}{2}\Paren{(1+3\dst)^j(1-3\dst)^\ell+(1-3\dst)^j(1+3\dst)^\ell}
\end{equation}
\emph{(Compare this to Eq.~(3.22)!)}. Plugging this into~\cref{eq:analogue:320} leads to
\begin{align}
	\mutualinfo{\mathbf{b}}{X} 
	&\leq \frac{\ab}{4} \sum_{j=0}^\infty\sum_{\ell=0}^\infty e^{-\frac{2\ns}{\ab}} \frac{(\ns/\ab)^{j+\ell}}{j!\ell!}\Paren{1- \tfrac{(1+3\dst)^j(1-3\dst)^\ell+(1-3\dst)^j(1+3\dst)^\ell}{2} }^2 \notag\\
	&=  \frac{\ab}{2} \sinh^2 \frac{9\ns \dst^2}{\ab} \notag\\
	&\leq \frac{81\ns^2 \dst^4}{2\ab} \cdot e^{\frac{3\ns \dst^2}{\ab}}
\end{align}
the last equality again following either by somewhat tedious series computations, or a symbolic computation system such as Julia, Mathematica, or Maple; and the inequality being $\sinh u \leq u e^{u^2/6}$ (for $u\in\R$.)\footnote{See, \eg \url{https://math.stackexchange.com/a/1759409/75808}.} We can then conclude as in Section~3.2: since we must have $\mutualinfo{\mathbf{b}}{X}  \gtrsim 1$ by Fact~3.1, we must have $\frac{\ns^2 \dst^4}{\ab} \gtrsim 1$, showing the $\ns = \bigOmega{\sqrt{\ab}/\dst^2}$ lower bound.
\end{solution}

\begin{question}\label{ex:2/3:lb:applications}
Verify that applying Theorem~3.9 to (i)~the uniform distribution $\uniform_\ab$ and (ii)~the ``Zipf'' distribution $\q\in\distribs{\ab}$ such that $\q(i) \propto 1/\sqrt{i}$ leads, in both cases, to an $\Omega(\sqrt{\ab}/\dst^2)$ sample complexity lower bound for identity testing.
\end{question}
\begin{solution}
(i)~When applying Theorem~3.9 to the uniform distribution (\ie $\q=\uniformOn{\ab}$), for any given $\dst \in (0,1/4)$ the vector $\tilde{\q}^{-\max}_{-4\dst}$ is the $m$-dimensional vector with all coordinates equal to $1/\ab$, where $m \eqdef \ab-(1+\flr{4\dst\ab}) \geq (1-4\dst-1/\ab)\ab$ (since we removed the first coordinate, as well as the last $\flr{4\dst\ab}$). Then,
\[
	\norm{\tilde{\q}^{-\max}_{-4\dst}}_{2/3} = \Paren{m \cdot \frac{1}{\ab^{2/3}}}^{3/2} = \frac{m^{3/2}}{\ab}
	\geq (1/2-4\dst)^{3/2} \sqrt{\ab} \geq \frac{\sqrt{\ab}}{4^{3/2}}
\]
the first inequality as $\ab \geq 2$, and the second for $\dst \leq 1/16$. This gives the lower bound $\bigOmega{\norm{\tilde{\q}^{-\max}_{-4\dst}}_{2/3}/\dst^2} = \bigOmega{\sqrt{\ab}/\dst^2}$ we wanted.\medskip

(ii)~When applying the theorem with $\q$ set to be the Zipf distribution, that is, $\q(i) = \frac{1}{H_{\ab, 1/2}\sqrt{i}}$ for every $i\in [\ab]$ with $H_{\ab, 1/2}  = \sum_{i=1}^\ab \frac{1}{\sqrt{i}} = \bigTheta{\sqrt{\ab}}$, for any given $\dst\in(0,1/4)$ we get
\[
	\norm{\tilde{\q}^{-\max}_{-4\dst}}_{2/3}
	= H_{\ab, 1/2}^{-1}  \Paren{ \sum_{i=2}^{\ell} \frac{1}{i^{1/3}} }^{3/2} \asymp \frac{\Paren{ \ell^{2/3} }^{3/2}}{H_{\ab, 1/2}}
	\asymp \frac{\ell}{\sqrt{\ab}}
\]
where $\ell$ is the smallest integer such that $H_{\ab, 1/2}^{-1}\sum_{i=\ell}^\ab 1/\sqrt{i} \leq 4\dst$. Computing the asymptotics of the sum shows that $\ell$ satisfies $\sqrt{\ab}-\sqrt{\ell} \sim 4\dst\sqrt{\ab}$, leading to $\ell \sim (1-4\dst)^2 \ab$. For $\dst \leq 1/5$ (for instance), this again gives $\norm{\tilde{\q}^{-\max}_{-4\dst}}_{2/3} = \bigTheta{\sqrt{\ab}}$, and Theorem~3.9 then yields the same (tight, in view of the identity testing upper bound) sample complexity lower bound $\bigOmega{\sqrt{\ab}/\dst^2}$.
\end{solution}

\begin{question}\label{exo:reexpress:algos:frequency:only}
Check that you can express several of the algorithms in Section~2.1 as a function of $\freq$ only (as defined in Section~3.3). Specifically, verify this for Algorithms~1,~2 and~4. Verify this also for Algorithm~3, keeping in mind that this algorithm was stated and analyzed in the Poissonized setting: what does it change?
\end{question}
\begin{solution}
We can rewrite $Z_1$, from Algorithm~1, as
\begin{align*}
	Z_1  &= \frac{1}{\binom{\ns}{2}} \sum_{i\in\domain} \binom{\occur_i}{2} 
		= \frac{1}{\binom{\ns}{2}} \sum_{i\in\domain} \sum_{j=0}^{\ns} \binom{j}{2} \indic{\occur_i=j}
		= \frac{1}{\binom{\ns}{2}} \sum_{j=0}^{\ns} \binom{j}{2} \sum_{i\in\domain}\indic{\occur_i=j} \\
		&= \frac{1}{\binom{\ns}{2}} \sum_{j=0}^{\ns} \binom{j}{2} \freq_j\,.
\end{align*}
Similarly, $Z_2$, from Algorithm~2, is given by
\[
	Z_2 = \frac{1}{\ns} \freq_1
\]
while $Z_4$, from Algorithm~4, can be written as
\[
	Z_4 = \frac{1}{2}\sum_{j=0}^{\ns} \abs{\frac{j}{\ns}-\frac{1}{\ab}} \freq_j\,.
\]
We can also rewrite $Z_3$ as a function of the fingerprint $\freq$; however, in the Poissonized setting, $\freq\in \N^{\N}$ (it is no longer a finite-dimensional vector, but a sequence) and does not necessarily sum to $\ns$ anymore. With this in mind, we get
\begin{align*}
	Z_3 
	&= \sum_{i\in\domain} \frac{\Paren{\occur_i-\ns/\ab}^2-\occur_i}{\ns/\ab} \\
	&= \sum_{i\in\domain} \sum_{j=0}^{\infty} \frac{\Paren{j-\ns/\ab}^2-j}{\ns/\ab} \indic{\occur_i=j} \\
	&= \sum_{j=0}^{\infty} \frac{\Paren{j-\ns/\ab}^2-j}{\ns/\ab} \freq_j\,.
\end{align*}
Note that we now have to sum over all integers, not just up to $\ns$.
\end{solution}

\begin{question}[$\star$]\label{ex:reduction:monotone}
  Prove that the mapping $\Phi$ defined in Eq.~(3.40) does satisfy the requirements of a reduction, for $\ab'=2\ab$ and $\dst'=\dst/2$. That is, if $\p\in\distribs{\ab}$ is $\dst$-far from $\uniform_\ab$, then $\Phi(\p)\in\distribs{2\ab}$ is $\dst'$-far from every distribution $\q\in\property^{\searrow}_{2\ab}$. \textit{(Hint: for any given monotone $\q$, analyse the distance $\totalvardist{\Phi(\p)}{\q}$ according to whether $\q(\ab) > 1/(2\ab)$ or not, relating this to the set $S\subseteq[\ab]$ on which $\p$ is greater than $\uniform_\ab$.)} Moreover, show that this loss by a factor $1/2$ in the distance is necessary.
\end{question}
\begin{solution}
Fix any $\p\in\distribs{\ab}$ such that $\totalvardist{\p}{\uniform_\ab}>\dst$, and let $S\subseteq [\ab]$ the set which witnesses it:
$S\eqdef \setOfSuchThat{ 1\leq i\leq \ab }{\p(i)> 1/\ab }$ 
which satisfies 
\[
	\p(S)-\uniform_\ab(S) = \uniform_\ab([\ab]\setminus S)- \p([\ab]\setminus S) >\dst
\]
Further define $T \eqdef [\ab]\setminus S \subseteq \{1,2,\dots,\ab\}$, and $S+\ab = \setOfSuchThat{i+\ab}{i\in S} \subseteq \{\ab+1,\dots,2\ab\}$. 
By definition of $\Phi(\p)$, we then have
\[
	\uniform_{2\ab}(T)-\Phi(\p)(T)=\Phi(\p)(S+\ab)-\uniform_{2\ab}(S+\ab)= \frac{1}{2}(\p(S)-\uniform_\ab(S)>\frac{\dst}{2}\,.
\]
Now, fix any monotone distribution $\q\in\property^{\searrow}_{2\ab}$. We have two cases:
\begin{itemize}
\item If $\q(\ab) > 1/(2\ab)$, then, since $\q$ is monotone, $\q(i) > 1/(2\ab)$ for every $i \leq \ab$. This implies
\[
	\q(T)-\Phi(\p)(T) \geq \uniform_{2\ab}(T)-\Phi(\p)(T) > \frac{\dst}{2}
\]
\item If $\q(\ab) \leq 1/(2\ab)$, then $\q(i) \leq 1/(2\ab)$ for every $i \geq \ab$. This implies
\[
	\Phi(\p)(S+\ab)-\q(S+\ab) \geq \Phi(\p)(S+\ab)-\uniform_{2\ab}(S+\ab) > \frac{\dst}{2}
\]
\end{itemize}
This shows that $\totalvardist{\Phi(\p)}{\q} > \dst/2$. For the ``necessary'' part, consider $\p$ such that $\p(1)=1$.
\end{solution}

%%%%%%%%%%%%%%%%%%%%%%%%%%%%%%%%%%%%%%%%%%%%%%%%%%%%%%%%%%%%%%%%%%%%%%%%%%%%%%
\newcommand{\pbd}{\hspace{1pt}{\cdot}\kern -4pt {\subset}\kern -3pt {\rtimes}}
%%%%%%%%%%%%%%%%%%%%%%%%%%%%%%%%%%%%%%%%%%%%%%%%%%%%%%%%%%%%%%%%%%%%%%%%%%%%%%
\begin{question}\label{ex:testing:pbd}
  A \emph{Poisson Binomial Distribution} (PBD) with parameters $\ab$ and $\vec{p}=(p_1,\dots,p_\ab)$ is the distribution of the sum of $\ab$ independent Bernoulli random variables $X_1,\dots,X_\ab$, where $X_i \sim \bernoulli{p_i}$. (This is a generalization of Binomial distributions, which correspond to $p_1=\dots=p_\ab$.) Let $\property^{\pbd}_{\ab}$ denote the class of all PBDs with parameter $\ab$. Using the facts that (1)~$\property^{\pbd}_{\ab}$ can be agnostically learned with $O(\log^2(1/\dst)/\dst^2)$ samples (independent of $\ab$)~\citep{DDS:PBD:15}, and (2)~the ``standard'' Binomial distribution $\binomial{\ab}{1/2}$ is a PBD, show that testing $\property^{\pbd}_{\ab}$ has sample complexity $\Omega(\ab^{1/4}/\dst^2)$ (as long as $\dst \geq 1/2^{O(\ab^{1/8})}$). \textit{(Hint: combine the results of Sections~3.4 and~3.5.)}
\end{question}
\begin{solution}
For convenience, denote by $\q$ the Binomial $\binomial{\ab}{1/2}$, and fix $\errprob = 1/3$. We will apply Theorem~3.11 with $\property' \eqdef \{\q\}$ and $\property \eqdef \property^{\pbd}_{\ab}$, so that $\property' \subseteq \property$.

From the statement of the exercise, from~\citep{DDS:PBD:15}\footnote{This is not explicit in the paper, but their cover-based approach implies agnostic learning, for a (large, but constant) $C\geq 1$.} we know that the sample complexity of agnostically learning $\property^{\pbd}_{\ab}$ is
\[
	\ns_{\mathcal{L}}(\ab,\dst,1/3) = \bigO{\frac{\log^2(1/\dst)}{\dst^2}}
\]
so to apply the theorem it remains to give a lower bound on $\ns_{\mathcal{T}}(\ab,\dst,1/3)$ (sample complexity of testing $\property'$) and comparing the two. But testing our $\property'$ is by definition testing identity to $\q$, for which we can apply Theorem~3.9: for all $\dst\in(0,1/4)$, 
\[
		\ns_{\mathcal{T}}(\ab,\dst,1/3) = \bigOmega{\norm{\tilde{\q}^{-\max}_{-4\dst}}_{2/3}/\dst^2}\,,
\]
where $\tilde{\q}^{-\max}_{-4\dst}$ is as defined in the statement of Theorem~3.9. In our case, to get a hold of 
$
	\norm{\tilde{\q}^{-\max}_{-4\dst}}_{2/3}
$, we will make three observations. \emph{(The argument below is heuristic and informal, but can be made rigorous).} First, the maximum probability value of $\q$ is $\frac{1}{2^n}\binom{\ab}{\ab/2} = \bigTheta{\frac{1}{\sqrt{\ab}}}$, so removing it will be negligible. Second, to remove the $4\dst$ probability mass from the tails of $\q$, we can use Hoeffding's inequality (Corollary~A.4) as heuristic, since even though it provides an upper bound instead of a lower bound, Hoeffding's inequality is essentially tight for the standard Binomial. Working it out, this then tells us that
\[
	\bPr{ \abs{X - \frac{\ab}{2}} \geq m } \approx 4\dst 
\]
for $m \asymp \sqrt{\ab \log\frac{1}{\dst}}$, which means that we only keep $2m = \bigTheta{\sqrt{\ab \log\frac{1}{\dst}}}$ coordinates in our vector $\tilde{\q}^{-\max}_{-4\dst}$ (since $\q$ is symmetric around its expectation, those are the $2m$ middle elements of the domain, centered at $\ab/2$). That is quite difficult to bound, though, so since we are only aiming to \emph{lower bound} $\norm{\tilde{\q}^{-\max}_{-4\dst}}_{2/3}$ we can remove more elements, and only keep the middle $2m'$ for say $m' \eqdef \sqrt{\ab} \leq m$.


Finally, since $\q$ is unimodal, among those $2\sqrt{\ab}$ elements the minimum probability is $\q(\ab/2+\sqrt{\ab})$, which by Stirling can be bounded as
\[
	\q(\ab/2+m') = \frac{1}{2^\ab} \binom{\ab}{\frac{\ab}{2}+\sqrt{\ab}} \asymp \frac{1}{\sqrt{\ab}}  \,.
\]
Thus, we can bound
\begin{align*}
	\norm{\tilde{\q}^{-\max}_{-4\dst}}_{2/3} 
	&\geq \Paren{ 2m'\cdot \q(\ab/2+m')^{2/3} }^{3/2}  \\
	&\asymp \ab^{3/4} \cdot \q(\ab/2+\sqrt{\ab})  \\
	&\asymp \ab^{1/4}
\end{align*}
and Theorem~3.9 then tells us that $\ns_{\mathcal{T}}(\ab,\dst,1/3) = \bigOmega{\ab^{1/4}/\dst^2}$. \emph{(Intuitively: the standard Binomial distribution is ``roughly flat'' within a few standard deviations of its expectation, which means it is sort-of-uniform on a domain of size $\Theta(\sqrt{\ab})$. So we get ``the uniformity testing lower bound'' but on a domain of size $\Theta(\sqrt{\ab})$, not $\ab$: hence the $\ab^{1/4}$.}

To conclude and apply Theorem~3.11, we just need to check the third item, and see if there is a range of parameters for which
\[
		\ns_{\mathcal{L}}(\ab,C\dst,1/3)  \leq \frac{1}{2}\ns_{\mathcal{T}}(\ab,3C\dst,1/3) 
\]
(where again, $C\geq 1$ is a constant, implicit in~\citep{DDS:PBD:15}). This boils down to checking when
\[
		\frac{\log^2(1/\dst)}{\dst^2} \ll \frac{\ab^{1/4}}{\dst^2}
\]
which is true for $\dst \geq 1/2^{O(\ab^{1/8})}$.  Theorem~3.11 then yields the $\Omega(\ab^{1/4}/\dst^2)$ lower bound for testing $\property^{\pbd}_{\ab}$ in that parameter regime. 
\end{solution}

%%%%%%%%%%%%%%%%%%%%%%%%%%%%%%%%%%%%%%%%%%%%%%%%%%%%%%%%%%%%%%
%%%%%%%%%%%%%%%%%%%%%%%%%%%%%%%%%%%%%%%%%%%%%%%%%%%%%%%%%%%%%%
\chapter{Testing with Constrained Measurements}
  \label{chap:constrained}
  
  \begin{question}\label{exo:error:amplification}
Verify that the error amplification technique discussed in Lemma~1.1 still goes through in the communication-constrained distributed setting.
\end{question}
\begin{solution}[Sketch]
Regardless of the specific distributed setting (private-coin, public-coin, or sequentially interactive), one can still run $m=\bigO{\log(1/\errprob)}$ independent instances of the protocol on $m$ disjoint sets of users, and use the same argument as in~Lemma~1.1.
\end{solution}

\begin{question}\label{exo:identity:reduction}
Verify that the reduction from identity to uniformity testing discussed in Section~2.2.3 still goes through in the communication-constrained distributed setting, both in the private- and public-coin settings. Do the users need to know the reference distribution $\q$?
\end{question}
\begin{solution}
As described in \citet[Proposition~A.16]{AcharyaCT19b}, the reduction does go through, as each user can ``locally'' and independently apply the mapping $\Psi_\q$ to their sample, to get a draw from the distribution $\Phi_\q(\p)$. This preserves the setting of randomness (private- or public-coin), as it only require private randomness; however, as mentioned in~\citet[Remark~A.17]{AcharyaCT19b}, all users \emph{do} need to know the reference distribution $\q$ (in order to be able to compute the mapping~$\Psi_\q$.

However, the two specific approaches seen in Section~4, for private-coin protocols (Section~4.2) and public-coin protocols (Section~4.3) can be used to perform identity testing \emph{even if the users do not know the reference $\q$}. Namely:
\begin{itemize}
	\item In the private-coin case, distributed simulation (Theorem~4.2) can still be used to simulate $\ns'\asymp \ns 2^\numbits/\ab$ samples from $\p$ at the server. The server (which \emph{does} know $\q$, even if it is the only one to do so) can then convert these $\ns'$ \iid samples into $\ns'$ \iid samples from $\Phi_\q(\p)$, performing the reduction to uniformity testing in a centralized fashion.
	\item In the public-coin case, domain compression (Theorem~2.12) is used to convert the $\ns$ \iid samples from $\p$ into $\ns$ \iid samples from some (randomly chosen) $\p_{\Pi}$ on a domain of size $L \eqdef 2^\numbits$. This still works and can be done without knowledge of $\q$; then, the guarantee of Theorem~2.12 ensures that $\totalvardist{\p_{\Pi}}{\q_{\Pi}} \gtrsim \sqrt{{L}/{\ab}}\cdot\totalvardist{\p}{\q}$. This means that the server (which knows $\q$ as well as the public randomness used to choose $\Pi$, and thus can compute what the ``new reference'' $\q_{\Pi}$ is) can apply the reduction from identity testing (with reference $\q_{\Pi}$ over $[L]$) to uniformity testing locally, since it has all it needs for this: $\ns$ \iid samples from $\p_{\Pi}$, and knowledge of $\q_{\Pi}$ and $\dst' \asymp \sqrt{{L}/{\ab}}\cdot\dst$.
\end{itemize}
\end{solution}

\begin{question}[$\star$]\label{exo:simulate:infer}
Extend the argument of Lemma~4.3 to $\numbits\geq 1$, to establish the more general Theorem~4.2. \emph{(Hint: suppose that $2^\numbits -1$ divides $\ab$, and partition the domain in $m\eqdef \ab/(2^\numbits -1)$ sets. Each pair of users is now ``assigned'' one of these sets.)}
\end{question}
\begin{solution}
As suggested, partition the domain into $m\eqdef \clg{\ab/2^\numbits -1}$ sets (\eg intervals) $S_1,\dots, S_m$, each of size $\ab' \eqdef 2^\numbits -1$ except at most one possibly smaller (if $2^\numbits -1$ does not divide $\ab$). As in the proof of Lemma~4.3, we first show how to generate one sample from $2m$ users.

Divide these $2m$ users into $m$ pairs, where users of the pair $(2i-1, 2i)$ are ``assigned'' set $S_i$. The $\numbits$-bit message these two users send is defined as follows: either the all-zero sequence $\textbf{0}_\numbits$ if their sample did not fall in $S_i$, or, if it did, the exact value of the sample (they can do so, as $|S_i| \leq 2^\numbits -1$, as long as the protocol specified an encoding beforehand):
\begin{align*}
	Y_{2i-1} &= 
		\begin{cases}
			\textrm{encode}(X_{2i-1}) &\text{ if } X_{2i-1} \in S_i\\
			\textbf{0}_\numbits &\text{ otherwise.}
		\end{cases} \\
	Y_{2i} &= 
		\begin{cases}
			\textrm{encode}(X_{2i}) &\text{ if } X_{2i} \in S_i\\
			\textbf{0}_\numbits &\text{ otherwise.}
		\end{cases}
\end{align*}
As in the single-bit ($\numbits=1$) case, the server, upon receiving these $2m$ messages, will check the following
two conditions:
\begin{itemize}
	\item there exists one, and only one, pair $(2i - 1, 2i)$ of users for which the ``even'' user sent a non-zero value (that is, $Y_{2i}\neq  \textbf{0}_\numbits$); and
	\item for this pair $(2i - 1, 2i)$, the ``odd'' user sent the all-zero sequence ($Y_{2i}=\textbf{0}_\numbits$).
\end{itemize}
If those two conditions do not simultaneously hold, then the server aborts (does not output any sample, but instead the special symbol $\bot$). Otherwise, the server outputs (the decoding of) $Y_{2i}$, which is $X_{2i}$, as its sample.
Then, analogously to the proof of Lemma~4.3, for any $j\in[\ab]$, letting $i(j)$ be the index of the set such that $j\in S_{i(j)}$, the probability to output $j$ is
\[
	\bPr{\text{output is } j } = \p(j)\cdot (1-\p(S_{i(j)})) \prod_{\substack{1\leq i\leq m\\ i\neq i(j)}} (1-\p(S_i)) = \p(j) \prod_{i=1}^m (1-\p(S_i))\,.
\]
We have $\bPr{\text{output is } j }\propto \p(j)$ for all $j\in[\ab]$, so all we need to prove, as before, is that $\bPr{\text{output is } \bot }$ is not too close to one, or, equivalently, that $\prod_{i=1}^m (1-\p(S_i))$ is not vanishingly small. The exact same argument as the one leading to Eq.~(4.4) shows that 
\[
	\prod_{i=1}^m (1-\p(S_i)) \geq \frac{1}{4}
\]
as long as $\max_{1\leq i\leq m} \p(S_i) \leq \frac{1}{2}$, which is implied by $\norminf{\p} \leq 1/2$ and thus can be ensured by using the same trick as in Lemma~4.3 (only losing, as we did there, a factor 2 in the number of users). Altogether, this establishes that we can generate (on expectation) \emph{one} sample from $\p$ using the $\numbits$-bits messages from
\[
		4 \cdot 4m = 16\clg{\frac{\ab}{2^\numbits -1}}
\]
users. By repeating this on disjoint groups of users, this yields Theorem~4.2, showing that we can generate an expected
\[
	\ns' \geq \frac{1}{16}\cdot \frac{\ns}{\clg{\frac{\ab}{2^\numbits -1}}} \asymp \frac{2^\numbits\ns}{\ab}
\]
\iid samples, using the $\numbits$-bits messages from $\ns$ users. For a more detailed proof and discussion of~Theorem~4.2, see~\citet[Theorem~IV.9]{AcharyaCT19b}.
\end{solution}

\begin{question}[$\star\star$]\label{exo:heterogeneous}
Extend the argument of Theorem~4.2 further to apply to the case where user has a communication constraint $\numbits_i$ (heterogeneous constraints among users). Establish an analogous bound, with $2^\numbits$ replaced by $\frac{1}{\ns}\sum_{j=1}^\ns 2^{\numbits_j}$. \emph{(Hint: consider a dyadic partition of the domain $[\ab]$. It \emph{should} work.)}
\end{question}
\begin{solution}\itshape
No solution for this one (for now at least). Feel free to contact me if you've tried and are stuck!
\end{solution}
  
  
\printbibliography

\end{document}
