% !Mode:: "TeX:DE:UTF-8:Main"
%
%
%JOURNAL CODE  SEE DOCUMENTATION

\documentclass[biber]{nowfnt} % creates the journal version, needs biber version
%wrapper for book and ebook are created automatically.

\usepackage[utf8]{inputenc}

\def\withcolors{1}
\def\withnotes{1}
\usepackage{ccanonne}

%ARTICLE TITLE
\title{Distribution Testing, From Stats to CS (and Back?)}


%ARTICLE SUB-TITLE
\subtitle{Just a working title}


%AUTHORS FOR COVER PAGE 
% separate authors by \and, item by \\
% Don't use verbatim or problematic symbols.
% _ in mail address should be entered as \_
% Pay attention to large mail-addresses ...

%if there are many author twocolumn mode can be activated.
%\booltrue{authortwocolumn} %SEE DOCUMENTATION
\maintitleauthorlist{
Cl\'ement L. Canonne \\
University of Sydney\\
clement.canonne@sydney.edu.au
}

%ISSUE DATA AS PROVIDED BY NOW
\issuesetup
{%
 copyrightowner={A.~Heezemans and M.~Casey},
 volume        = xx,
 issue         = xx,
 pubyear       = 2018,
 isbn          = xxx-x-xxxxx-xxx-x,
 eisbn         = xxx-x-xxxxx-xxx-x,
 doi           = 10.1561/XXXXXXXXX,
 firstpage     = 1, %Explain
 lastpage      = 18
 }

%BIBLIOGRAPHY FILE
%\addbibresource{sample.bib}
%\addbibresource{testfnt.bib}
\addbibresource{references.bib}

% \usepackage{mwe}

%AUTHORS FOR ABSTRACT PAGE
\author[1]{Cl\'ement L. Canonne}

\affil[1]{University of Sydney; clement.canonne@sydney.edu.au}

\articledatabox{\nowfntstandardcitation}

\begin{document}

\makeabstracttitle

\begin{abstract}
We focus on some specific problems in distribution testing,
taking goodness-of-fit as a running example. In particular,
we do not aim to provide a comprehensive summary of all
the topics and results in the area; but will provide self-
contained proofs and derivations of the main results, trying
to highlight the unifying techniques.
\end{abstract}

%%%%%%%%%%%%%%%%%%%%%%%%%%%%%%%%%%%%%%%%%%%%%%%%%%%%%%%%%%%%%%
%%%%%%%%%%%%%%%%%%%%%%%%%%%%%%%%%%%%%%%%%%%%%%%%%%%%%%%%%%%%%%
\chapter{What is distribution testing?}

\section{Formulation, and relation to Hypothesis Testing}
\section{Why total variation distance?}
\section{The road not taken: tolerant testing}
\section{Historical notes}


%%%%%%%%%%%%%%%%%%%%%%%%%%%%%%%%%%%%%%%%%%%%%%%%%%%%%%%%%%%%%%
%%%%%%%%%%%%%%%%%%%%%%%%%%%%%%%%%%%%%%%%%%%%%%%%%%%%%%%%%%%%%%
\chapter{Testing goodness-of-fit of univariate distributions}

(In-depth chapter with derivation of many results. Focus on one-sample
testing.)

\section{Uniformity testing}
Thorough presentation of the many algorithms for uniformity testing,
with proofs, and relative advantages of each.

\section{Identity testing}
Discuss the instance-optimal setting as well, and the relation to unifor-
mity testing.


%%%%%%%%%%%%%%%%%%%%%%%%%%%%%%%%%%%%%%%%%%%%%%%%%%%%%%%%%%%%%%
%%%%%%%%%%%%%%%%%%%%%%%%%%%%%%%%%%%%%%%%%%%%%%%%%%%%%%%%%%%%%%
\chapter{General algorithmic frameworks}
\section{Testing-by-learning}
\section{Shape restrictions}
\section{Fourier approach}
\section{The $\lp[2]$ testing framework of Diakonikolas--Kane}

%%%%%%%%%%%%%%%%%%%%%%%%%%%%%%%%%%%%%%%%%%%%%%%%%%%%%%%%%%%%%%
%%%%%%%%%%%%%%%%%%%%%%%%%%%%%%%%%%%%%%%%%%%%%%%%%%%%%%%%%%%%%%
\chapter{Testing high-dimensional distributions}

\section{Testing product Bernoulli and Gaussian distributions}
Mention the hardness of robust testing (Diakonikolas--Kane--Stewart),
to show the importance of the assumption.

\section{Bayesian networks}
\section{Ising models and Markov Random Fields}

%%%%%%%%%%%%%%%%%%%%%%%%%%%%%%%%%%%%%%%%%%%%%%%%%%%%%%%%%%%%%%
%%%%%%%%%%%%%%%%%%%%%%%%%%%%%%%%%%%%%%%%%%%%%%%%%%%%%%%%%%%%%%
\chapter{Testing with Constrained Measurements}
Discuss what those can be (local privacy, noisy channels, linear of
partial measurements, etc). Focus on communication constraints ($\numbits$-bit
measurements). Talk about adaptive schemes?
\section{Continuous distributions and quantization}
\section{Random hashing and domain compression}
\section{Application: distributed inference}

%%%%%%%%%%%%%%%%%%%%%%%%%%%%%%%%%%%%%%%%%%%%%%%%%%%%%%%%%%%%%%
%%%%%%%%%%%%%%%%%%%%%%%%%%%%%%%%%%%%%%%%%%%%%%%%%%%%%%%%%%%%%%
\chapter{Information-theoretic lower bounds}
\section{Proving hardness by reductions}
\section{Indistinguishability via moment-matching}
\section{Ingster's method}
\section{Application: distributed inference}

%%%%%%%%%%%%%%%%%%%%%%%%%%%%%%%%%%%%%%%%%%%%%%%%%%%%%%%%%%%%%%
%%%%%%%%%%%%%%%%%%%%%%%%%%%%%%%%%%%%%%%%%%%%%%%%%%%%%%%%%%%%%%
\chapter{Beyond \iid sampling}
Less detailed: This is a short chapter, where each section contains one key result/example
to illustrate the model or question, along with references.

\section{Conditional sampling}
Ability to sample, conditioning on specific events.

\section{Evaluation access}
Exact or approximate access to the probability mass or cumulative
distribution function.

\section{Testing properties of Markov chains}
Only one sample (trajectory) of a Markov chain.

\section{Causal models and interventions}


%%%%%%%%%%%%%%%%%%%%%%%%%%%%%%%%%%%%%%%%%%%%%%%%%%%%%%%%%%%%%%
%%%%%%%%%%%%%%%%%%%%%%%%%%%%%%%%%%%%%%%%%%%%%%%%%%%%%%%%%%%%%%
\chapter{Quantum distribution testing}

Less detailed: Not many full derivations here, mostly statement of the results and
sketch of the proofs/ideas. Focus on testing the fully mixed state.
\section{What is quantum distribution testing?}
Brief (very) intro to quantum computing; mention of different models
(treated below), and their respective advantages/disadvantages.
\section{Testing classical distributions with quantum measurements}
Short section. This is mostly about getting faster algorithms, discuss
the (lack of) implications for sample complexity.
\section{Testing quantum states\dots}
\subsection{With local measurements}
\subsection{With unentangled measurements}
Two flavors: adaptive and non-adaptive. Discuss the recent work of
Bubeck, Chen, and Li showing a separation between the two.
\subsection{With entangled measurements}
Tight bounds for testing mixed states: in particular, O’Donnell and
Wright.

\appendix

%BACKMATTER SEE DOCUMENTATION
\backmatter  % references, restarts sample

\printbibliography

\end{document}
